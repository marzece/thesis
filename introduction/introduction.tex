
%: ----------------------- introduction file header -----------------------
\chapter{Introduction}

% the code below specifies where the figures are stored
\ifpdf
    \graphicspath{{introduction/figures/PNG/}{introduction/figures/PDF/}{introduction/figures/}}
\else
    \graphicspath{{introduction/figures/EPS/}{introduction/figures/}}
\fi

% ----------------------------------------------------------------------
% ----------------------- introduction content -------------------------
% ----------------------------------------------------------------------
This thesis discusses two seperate results on the topic of solar neutrinos.
The first is the development and evaluation of a novel model for neutrino
mixing that suggests neutrinos might mix differently in areas of near zero
matter density.
The second is a measurement of the $\ce{^{8}B}$ solar neutrino flux
in the SNO+ water phase.

Predictions and measurements associated with solar neutrinos, among other
sources, has provided strong evidence that neutrinos are massive and undergo
flavor oscillations.
The neutrino occupies a fairly unique role in modern particle physics.
Despite being ubiquitous, created by a wide variety of natural and human
designed sources,
many of its fundamental properties remain unknown.
The core reason for this uncertainty is that the neutrino interacts extremely rarely
with standard matter, meaning that a wide variety of techniques used
is difficult to study.

This chapter provides an overview of the history, physics and experimental
results that are most releveant to this work.
C
Chapter~\ref{ch:physics} provides a basic introduction to the physics associated
with neutrinos, neutrino interactions, and neutrino mixing.
These physical ideas are applied to solar physics and solar neutrinos in Chapter~\ref{ch:solarneuts}
to develop predictions for solar neutrino detection.
Chapter~\ref{ch:snop} details the SNO+ detector, its detection methods and
capabilities, and the upgrades from SNO to SNO+.
In Chapter~\ref{ch:chameleons} I describe a novel theory for neutrino oscillation
that modifies how neutrinos oscillate in areas of very low matter density,
and gives limitations on that theory from existing measurements.
Chapter~\ref{ch:solar_analysis} describes how data from the SNO+ detector
was used and analyzed to measure the solar neutrino flux;
Chapter~\ref{ch:results} describes results of that analysis.
Finally Chapter~\ref{ch:conclusions} provides summary and conclusion.
