
%: ----------------------- introduction file header -----------------------
\chapter{Introduction}

% the code below specifies where the figures are stored
\ifpdf
    \graphicspath{{introduction/figures/PNG/}{introduction/figures/PDF/}{introduction/figures/}}
\else
    \graphicspath{{introduction/figures/EPS/}{introduction/figures/}}
\fi

% ----------------------------------------------------------------------
% ----------------------- introduction content -------------------------
% ----------------------------------------------------------------------

\section{Neutrinos}
Neutrinos were first hyptothesized by Wolfgang Pauli in 1930.
The motivation for the proposal the apparent violation of energy
conservation in $\beta$ decay \citep{pauli_letter}.
Several years after Pauli's speculative proposal Enrico Fermi offered
a thorough model of beta decay that conserved energy using the neutrino
\citep{fermi_beta_decay}.
Fermi's model predicted such a small cross-section for the neutrino that some
doubted it would ever be observed \citep{bethe_impossible_to_observe}.
However, roughly two decades after its initial proposal, Frederick Reines \&
Clyde Cowan performed an experiment that involved bombarding a tank of cadmium
doped water with anti-neutrinos from nuclear reactor.
Doing this they were able to observe the rate and energy of inverse $\beta$
decays that ocurred.
The results were consitent with Fermi's model of $\beta$ decay and were
considered a confirmation of the neutrino's existence.

\subsection{Neutrino Flavor}
The first experimental evidence for neutrino flavor came in 1962 from an
experiment \citep{lederman_muon_flavor} that studied the interactions of
neutrinos that came from muon decay, and the interactions of neutrinos
from beta decay.
The experiment observed that neutrinos from came from muon decay would produce
muons upon interacting in a detector.
And neutrinos produced from $\beta$ decay would create electrons in the
detector.
This lead to the conclusion that there are two different varieties of neutrino,
the $\nu_e$ and the $\nu_{\mu}$, and the idea that lepton flavor is conserved.
The third lepton generation, the $\tau$ and the $\nu_{\tau}$ was discovered 13
years later in 1975 \citep{tau_discovery}.
% It's worth noting that lepton flavor conservation was not predicted or required
% by the standard model, for that reason it's known as an 'accidental symmetry'.
% ^^^^Not true??? Lepton number is an accidental symmetry, idk about lepton
% flavor Also the definition of accidental symmetry has to do with term in the
% lagrangian being to high dimensional (according to wikipedia) not what I
% said....though they might be the same somehow

\subsection{Neutrino Oscillations}
Neutrino oscillation is a result of the fact that neutrino flavors do not have
well defined masses, instead neutrino flavor states are composed of a mixture
of mass states, and vice-versa.
This can be stated more preciesly as
\begin{equation}
    \ket{\nu_{i}} = U_{i\ell}\ket{\nu_\ell}
\end{equation}
Where $\ket{\nu_\ell}$ represents the neutrino flavor states, $\ket{nu_i}$
represents the mass states, and $U_{i\ell}$ describes the mixing of these
states. $U_{i\ell}$ is known as the Pontecorvo–Maki–Nakagawa–Sakata (PMNS) matrix,
and it is exactly analogous to the Cabibbo–Kobayashi–Maskawa (CKM) matrix used
to describe quark mixing.
In the simplest case where the weak states and the mass states are the same
$U_{i\ell}$ would just be the identity matrix;
$U_{i\ell}$ must be unitary so that the probability of observing
a neutrino in any state is 1.

It is typical to characterize $U_{i\ell}$ with three angles
($\theta_{12}$, $\theta_{13}$, $theta_{23}$) and a complex
phase $\delta_{cp}$. Doing so allows for the SU(3) matrix to
be decomposed into 3 SU(2) matrices,
$$U_{12} =
\begin{bmatrix}
    \cos\theta_{12} & \sin\theta_{12} & 0  \\
    -\sin\theta_{12}& \cos\theta_{12} & 0  \\
    0 & 0 & 0  \\
\end{bmatrix}
$$,
$$
U_{13} =
\begin{bmatrix}
    \cos\theta_{13} & 0 & \sin\theta_{13}e^{-i\delta_{cp}}\\
    0 & 0 & 0  \\
    -\sin\theta_{13} e^{-i\delta_{cp}} & 0 & \cos\theta_{13}  \\
\end{bmatrix}
$$,
$$
U_{23} =
\begin{bmatrix}
    0 & 0 & 0  \\
    0 & \cos\theta_{23} & \sin\theta_{23} \\
    0 & -\sin\theta_{23} & \cos\theta_{23}   \\
\end{bmatrix}
$$.
These matrices can be multplied to produce the full mixing matrix
$U_{i\ell} = U_{23}U_{13}U_{12}$

The mixed nature of neutrino flavor and mass states gives rise to oscillations
in the flavor content of propagating neutrinos.
The Hamiltonian for a neutrino propagating in vaccume is
$H = -im\textbf{p}^2 $.
Using Schrodinger's equation,
$H\ket{\Psi}$ = $E\ket{\Psi}$
gives the differential equation
$\textbf{p} \ket{\Psi} = imE\ket{\Psi}$.
Applying the position operator to this yields
$\dfrac{d}{d\textbf{x}} \Psi(x) = imE\Psi(x)$.
Solving this equation for $\Psi(x)$,
$\Psi = e^{i\dfrac({E}{m}}$.

This shows that not only are neutrino flavor states
composed of mass states, that composition results
in flavor oscillation over time.

When neutrino propagate through matter this oscillation is altered.
The local density of other particles modifies the vaccume Hamiltonian,
adding a weak interaction potential.
This interaction comes from a neutral current interaction of the form
shown in $TODO$, or a flavor depedent charge current interction with the
leptons around the neutrino.
Since nearly all matter contains a much higher density of electrons
than the other flavors of charged lepton, the charged current reaction
modifies the electron neutrino potential and not the potential for the muon
or tau neutrion.
The result is that the electron density through which a neutrino propagates
can modify the effective mass-splitting and mixing angle for the electron
neutrino.
For a given neutrino energy $E_{\nu}$ there exists an electron density for which
the effective mixing angle is maximal, this is known as the resonant density.


The masses of the neutrino mass states are not known, there are limits placed
on the sum of the neutrino masses from observations of tritium decay, and
from astronomical considerations. % TODO add citation.


\begin{equation}
    H \ket{\nu_i} = m_i \ket{\nu_i}
\end{equation}

\subsection{Solar Neutrinos}
\subsubsection{The MSW Effect}
\subsection{Neutrino Experiments}
\subsubsection{Solar Experiments}
\subsubsection{Terrestrial Experiments}


