
%: ----------------------- introduction file header -----------------------
\chapter{Introduction}

% the code below specifies where the figures are stored
\ifpdf
    \graphicspath{{introduction/figures/PNG/}{introduction/figures/PDF/}{introduction/figures/}}
\else
    \graphicspath{{introduction/figures/EPS/}{introduction/figures/}}
\fi

% ----------------------------------------------------------------------
% ----------------------- introduction content -------------------------
% ----------------------------------------------------------------------

Predictions and measurements associated with solar neutrinos, among other
sources, has provided strong evidence that neutrinos are massive and undergo
flavor oscillations.
The neutrino occupies a fairly unique role in modern particle physics.
Despite being ubiquitous, created by a wide variety of physical phenomena,
many of its fundamental properties remain unknown.
The core reason for this uncertainty is that the neutrino interacts extremely rarely
with standard matter, meaning that a wide variety of techniques used
is difficult to study.

Predictions regarding the neutrino flux produced by the sun has been developed
since the 19XX~\cite{XXX}. Many refinements to these original predictions have
been made over the years as our understanding of the Sun and of nuclear reactions
have improved~\cite{XXX,XXX,XXX}.
Simultaneous with improvements in prediction, methods for detecting neutrinos
improved as well. The first experiment capable of successfully detecting
neutrinos from the Sun was the Homestake Neutrino Experiment.

Numerous experiments studying neutrinos from a wide variety of sources have
contributed over the last century to our current understanding of neutrino
properties.
These experiment have provided near conclusive evidence of neutrino flavor
oscillations, which in turn requires that the neutrino have mass, and that
lepton flavor is not a fundamental symmetry of the universe.

Chapter~\ref{ch:physics} provides a basic introduction to the physics associated
with neutrinos, neutrino interactions, and neutrino mixing.
These physical ideas are applied to solar physics and solar neutrinos in Chapter~\ref{ch:solarneuts}
to develop predictions for solar neutrino detection.
Chapter~\ref{ch:snop} details the SNO+ detector, its detection methods and
capabilities, and the upgrades from SNO to SNO+.
In Chapter~\ref{ch:chameleons} I describe a novel theory for neutrino oscillation
that modifies how neutrinos oscillate in areas of very low matter density,
and gives limitations on that theory from existing measurements.
Chapter~\ref{ch:solar_analysis} describes how data from the SNO+ detector
was used and analyzed to measure the solar neutrino flux;
Chapter~\ref{ch:results} describes results of that analysis.
Finally Chapter~\ref{ch:conclusions} provides summary and conclusion for
this work.
