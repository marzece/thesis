
%: ----------------------- introduction file header -----------------------
\chapter{Introduction}

% the code below specifies where the figures are stored
\ifpdf
    \graphicspath{{introduction/figures/PNG/}{introduction/figures/PDF/}{introduction/figures/}}
\else
    \graphicspath{{introduction/figures/EPS/}{introduction/figures/}}
\fi

% ----------------------------------------------------------------------
% ----------------------- introduction content -------------------------
% ----------------------------------------------------------------------

\section{Neutrinos}
Neutrinos were first hyptothesized by Wolfgang Pauli in 1930.
The motivation for the proposal the apparent violation of energy
conservation in $\beta$ decay \citep{pauli_letter}.
Several years after Pauli's speculative proposal Enrico Fermi offered
a thorough model of beta decay that conserved energy using the neutrino
\citep{fermi_beta_decay}.
Fermi's model predicted such a small cross-section for the neutrino that some
doubted it would ever be observed \citep{bethe_impossible_to_observe}.
However, roughly two decades after its initial proposal, Frederick Reines \&
Clyde Cowan performed an experiment that involved bombarding a tank of cadmium
doped water with anti-neutrinos from nuclear reactor.
Doing this they were able to observe the rate and energy of inverse $\beta$
decays that ocurred.
The results were consitent with Fermi's model of $\beta$ decay and were
considered a confirmation of the neutrino's existence.

\subsection{Neutrino Flavor}
The first experimental evidence for neutrino flavor came in 1962 from an
experiment \citep{lederman_muon_flavor} that studied the interactions of
neutrinos that came from muon decay, and the interactions of neutrinos
from beta decay.
The experiment observed that neutrinos from came from muon decay would produce
muons upon interacting in a detector.
And neutrinos produced from $\beta$ decay would create electrons in the
detector.
This lead to the conclusion that there are two different varieties of neutrino,
the $\nu_e$ and the $\nu_{\mu}$, and the idea that lepton flavor is conserved.
The third lepton generation, the $\tau$ and the $\nu_{\tau}$ was discovered 13
years later in 1975 \citep{tau_discovery}.
% It's worth noting that lepton flavor conservation was not predicted or required
% by the standard model, for that reason it's known as an 'accidental symmetry'.
% ^^^^Not true??? Lepton number is an accidental symmetry, idk about lepton
% flavor Also the definition of accidental symmetry has to do with term in the
% lagrangian being to high dimensional (according to wikipedia) not what I
% said....though they might be the same somehow

\subsection{Neutrino Oscillations}
Neutrino oscillation is a result of the fact that neutrino flavors do not have
well defined masses, instead neutrino flavor states are composed of a mixture
of mass states, and vice-versa.
This can be stated more preciesly as
\begin{equation}
    \ket{\nu_{i}} = U_{i\ell}\ket{\nu_\ell}
\end{equation}
Where $\ket{\nu_\ell}$ represents the neutrino flavor states, $\ket{nu_i}$
represents the mass states, and $U_{i\ell}$ describes the mixing of these
states. $U_{i\ell}$ is known as the Pontecorvo-Maki-Nakagawa-Sakata (PMNS) matrix,
and it is exactly analogous to the Cabibbo-Kobayashi-Maskawa (CKM) matrix used
to describe quark mixing.
In the simplest case where the weak states and the mass states are the same
$U_{i\ell}$ would just be the identity matrix;
Under the assumption that there are three flavor states and three mass states
$U_{i\ell}$ must be unitary so that the probability of observing
a neutrino in any state is 1.

It is typical to characterize $U_{i\ell}$ with three angles
($\theta_{12}$, $\theta_{13}$, $theta_{23}$) and a complex
phase $\delta_{cp}$. Doing so allows for the SU(3) matrix to
be decomposed into 3 SU(2) matrices,
$$U_{12} =
\begin{bmatrix}
    \cos\theta_{12} & \sin\theta_{12} & 0  \\
    -\sin\theta_{12}& \cos\theta_{12} & 0  \\
    0 & 0 & 0  \\
\end{bmatrix}
$$,
$$
U_{13} =
\begin{bmatrix}
    \cos\theta_{13} & 0 & \sin\theta_{13}e^{-i\delta_{cp}}\\
    0 & 0 & 0  \\
    -\sin\theta_{13} e^{-i\delta_{cp}} & 0 & \cos\theta_{13}  \\
\end{bmatrix}
$$,
$$
U_{23} =
\begin{bmatrix}
    0 & 0 & 0  \\
    0 & \cos\theta_{23} & \sin\theta_{23} \\
    0 & -\sin\theta_{23} & \cos\theta_{23}   \\
\end{bmatrix}
$$.
These matrices can be multplied to produce the full mixing matrix
$U_{i\ell} = U_{23}U_{13}U_{12}$

The mixed nature of neutrino flavor and mass states gives rise to oscillations
in the flavor content of propagating neutrinos.
The Hamiltonian for a neutrino propagating in vaccume is
$H = -im\textbf{p}^2 $.
Using Schrodinger's equation,
$H\ket{\Psi}$ = $E\ket{\Psi}$
gives the differential equation
$\textbf{p} \ket{\Psi} = imE\ket{\Psi}$.
Applying the position operator to this yields
$\dfrac{d}{d\textbf{x}} \Psi(x) = imE\Psi(x)$.
Solving this equation for $\Psi(x)$,
$\Psi = e^{i\dfrac({E}{m}}$.

This shows that not only are neutrino flavor states
composed of mass states, that composition results
in flavor oscillation over time and space.

When neutrino propagate through matter this oscillation is altered.
The local density of other particles modifies the vaccume Hamiltonian,
adding a weak interaction potential.
This interaction comes from a neutral current interaction of the form
shown in $TODO$, or a flavor depedent charge current interction with the
leptons around the neutrino.
Since nearly all matter contains a much higher density of electrons
than the other flavors of charged lepton, the charged current reaction
modifies the electron neutrino potential and not the potential for the muon
or tau neutrion.
The result is that the electron density through which a neutrino propagates
can modify the effective mass-splitting and mixing angle for the electron
neutrino.
For a given neutrino energy $E_{\nu}$ there exists an electron density for which
the effective mixing angle is maximal, this is known as the resonant density.

The masses of the neutrino mass states are not known, there are limits placed
on the sum of the neutrino masses from observations of tritium decay, and
from astronomical considerations. % TODO add citation.


\begin{equation}
    H \ket{\nu_i} = m_i \ket{\nu_i}
\end{equation}


\subsection{Neutrino Elastic Scattering}
\begin{figure}
    \begin{subfigure}[t]{0.53\textwidth}
        \centering
        \begin{feynman}
            \fermion[label=$e^{-}$]{0.00, 0.00}{1.00, 1.00}
            \fermion[label=$\nu_{e}$]{1.00, 1.00}{0.00, 2.00}
            \electroweak[label=$W^{+}$]{1.00, 1.00}{2.0, 1.00}
            \fermion[label=$\nu_{e}$, flip=true]{2.00, 1.00}{3.0, 0.00}
            \fermion[label=$e^{-}$, showArrow=true, flip=true]{2.0, 1.00}{3.00, 2.00}
        \end{feynman}
    \end{subfigure}
    \begin{subfigure}[t]{0.53\textwidth}
        \centering
        \begin{feynman}
            \fermion[label=$\nu_{x}$]{0.00, 0.00}{1.00, 1.00}
            \fermion[label=$\nu_{x}$]{1.00, 1.00}{0.00, 2.00}
            \electroweak[label=$Z$]{1.00, 1.00}{2.0, 1.00}
            \fermion[label=$e^{-}$, flip=true]{2.00, 1.00}{3.0, 0.00}
            \fermion[label=$e^{-}$, showArrow=true, flip=true]{2.0, 1.00}{3.00, 2.00}
        \end{feynman}
    \end{subfigure}
    \label{fig:feynman_es}
\end{figure}

Neutrino-electron elastic scattering (ES) is an important neutrino interaction
channel for this work. 
Figure~\ref{fig:feynman_es} shows the tree level feynman diagrams for the charged
current (CC) elastic scattering interaction~\ref{fig:feynman_es_cc} and the
neutral current (NC) elastic scattering interaction~\ref{fig:feynman_es_nc}.
The neutral current interaction is shown to involve $\nu_{\mathrm{x}}$ where $x=\mathrm{e, \mu, \tau}$,
and the charged current interaction involves only $\nu_{\mathrm{e}}$.






\subsection{Neutrino Experiments}
There's a long a diverse list of neutrino experiments that have contriubted to
our current understanding of neutrinos and neutrino oscillations.
I won't attempt to list them all here, but rather highlight the most immediatly
relevant to this work. A more compresive review can be found in \citep{FINDAREVIEW}.

\subsubsection{Homestake}
The first experiment to succesfully detect solar neutrinos was the Homestake
experiment.
The detector was composed of approximately 100000 gallons of dry cleaning fluid.
The choice of target was motivated by the high chlorine content in the cleaning
fluid. Neutrinos above an energy of XXX would interact with the chlorine via
beta decay, creating XXX. XXX would then decay to XXX with a half life of XXX.
Periodically $\ce{^{3}He}$ was bubbled through the
target liquid to extract the atoms of XXX that had been created. Once
extracted those atoms were observed with proportional counters to count the
number of XXX to XXX decays. The number of observed counts was proportional
to the solar neutrino interaction rate, and therefore the solar neutrino flux.

The homestake experiment ran from 1970 - 1990. %(XXX? is that right??)
The experiment was able to provide the first measurement of the solar neutrino
flux above XXX MeV. In 19XX they first reported a measure flux of
XXX, nearly a third of the expted rate which was XXX.
This deficiency became known as the solar neutrino problem, and it
was the first evidence for neutrino oscillation.
The deficiency was present across the entire lifetime of the Homestake experiment,
their final report flux was XXX.

\subsubsection{SNO}
The Sudbury Neutrino Observatory (SNO) is a water-cherenkov detector located
roughly $2$\,km underground near Sudbury Ontario in Canada, it ran from
$19XX$ to $20XX$.
SNO was primarily a solar neutrino detector, it had the unique benefit of
being able to detect neutrinos through three different interaction channels,
each channel had it's own sensitivity to different flavor neutrinos.
This allowed for a measurment of the $^8B$ solar neutrino flux that was not
dependent on the flavor composition of the incoming neutrinos.
This was accoplished by using a heavy-water ($\ce{^{2}H_{2}O}$) target.
Heavy-water's primary neutrino interactions are the
electron scattering interaction (ES), a charged current nuclear reaction (CC),
and a neutral current nuclear reaction (NC).
%TODO add feynman diagrams for those reactions.
There exists both charged current and a neutral current versions of the
ES interaction; since electron neutrinos can interact through either
of the two where as muon or tau neutrinos can only elastic scatter throug the
neutral current version, the ES cross-section for electron neutrinos is larger
than the cross-section for muon and tau neutrinos.
The difference in cross-section is energy dependent, but it's roughly a factor
of 6 for solar neutrino energies. The neutral current and charged current ES reactions
are not treated seperately because there is no detectable signature that
would allow you to discrimate between the two.

The NC interaction on a deuteron can break apart the neutron and proton
that comprises the nucleus. $\nu_{e} + D -> p + n + \nu_{e}$.
The free-neutron can then capture on the deuterium forming tritium ($\ce{^{3}H}$)
and emitting an $XXX$\,MeV gamma. %TODO how often does it capture on oxygen?????
The NC reaction has no neutrino flavor preference, so a measure of the rate
of NC rate along with the process' cross-section provides a flavor independent
measurement of the solar neutrino flux.

The charged current interaction on a deuteron converts a neutron to a
proton and produces an electron.$\nu_{e} + D -> H + H + e^{-}$.
This reaction can only occur when the charged lepton and the neutrino are the same
flavor. So for typical matter this reaction only occurs for the electron flavor
neutrinos, and so it provides a measurement of the electron flavor content
of the solar neutrino flux.

SNO was able to seperate and count the events of each type of interaction,
providing them three independent measurements of solar neutrinos. And
the rates of each measurement had a different dependence on the flavor content
of solar neutrinos.


\subsubsection{Super Kamiokande}
Super Kamiokande (SuperK) is a $XXX$ kton cylindrical water cherekov detector.
It started running in $19XX$ and has since made the most precise measurements of
atmospheric neutrinos and solar neutrinos so far.
It's the successor to the previous Kamiokande experiment, which was a significantly
smaller and had a higher energy threshold for detection.
SuperK can detect solar neutrino through only neutrino-electron elastic scattering,
they do not use a $\ce{D_{2}O}$ target and so are not sensitive to the
nuclear interactions that SNO used.
Their extremely large detector volume though provides them with far more exposure
than SNO could attain though. This results in a very precise measurement of the
elastic scattering rate.

The SuperK experiment seperates their dataset into 4 subsets, called
SuperK-I, SuperK-II, \textit{etc}. %TODO Roman numerals.
Each dataset covers several years of data taking.

\subsubsection{Borexino}
Borexino is $XXX$ kton spherical liquid scintillator detector. Their detector
apparatus is similar to that of a water-cherenkov detector, the significant difference
is that the water is replaced with pseudo-cumine, a liquid scintillator.
A charged particle moving through scintillator generates rougly 50-100 times
more light than a similar particle moving through just water.
Water-cherenkov detector are typically limited in energy threshold
and energy resolution by the number of photons produced and detected, a liquid scintillator
detector solves this problem.  Scintillation light, unlike cherenkov light,
is isotropic and provides no information about the direction the particle
was moving in.

Water-cherenkov detector are able to measure solar neutrinos by correlating the
direction of detected events with the position of the sun. Since Borexino is not
able to determine the direction of events within their detector, they instead
perform a spectroscopic measurement. The measurement requires
all sources of backgrounds to be accounted for and constrained from \textit{ex-situ}
measurements. Figure \ref{XXX} shows the observed spectrum by Borexino and the
spectrums of the constituent solar fluxes and backgrounds.

Borexino took data from 2007 to 2015(???), with a pause in 2010 to remove source
of radioactive backgrounds and improve the radio-purity of their detector.
With that data they've produced measurements of neutrino fluxes from the $\ce{^{7}Be}$,
$\ce{pep}$, $\ce{pp}$, and $\ce{^{8}B}$; they've also placed upper limits on the flux
of neutrinos from the CNO cycle and from the $hep$ solar reaction.
They're currently the only experiment to have measured the $\ce{pp}$ and $\ce{pep}$ neutrino
fluxes.

show spectrum and table of results

\subsubsection{KamLAND}
The Kamioka Liquid Scintillator Antineutrino Detector (KamLAND) is a liquid-scintillator experiment similar to Borexino.
It's primary physics goals were the detection of reactor anti-neutrinos, they are
however also sensitive to solar neutrinos.
Using analyses methods similar to Borexino they were able to determine the flux
of $\ce{7^{Be}}$.... %TODO....

Perhaps surpising is that KamLAND's reactor neutrino measurements are in some ways
more relevant to the study of solar neutrinos than their solar neutrino measurements.
The long baseline and low energy of reactor neutrinos that KamLAND detects gives
them unique senstivity to $\Delta m^{2}_{21}$. Other reactor neutrino experiments,
such as Daya Bay, RENO, \& Double Chooz are primarily senstive to neutrinos
with too short a baseline to be strongly affected by $\Delta m^{2}_{21}$.

As shown in figure XXX %TODO
the spectrum of reactor neutrinos that KamLAND detects is modified by an oscillatory
pattern that is determined primarily by $\Delta m^{2}_{21}$ and $\theta_{12}$.
By fitting for the amplitude and wavelength of those oscillations in the
spectrum values of $\Delta m^{2}_{21}$ and $\theta_{12}$ were determined
to be %TODO.
The value for $\Delta m^{2}_{21}$ is in disagreement with the value extracted
by solar experiments, although it cannot be ruled out that the disagreement
is a result of a statistical flucation. This discrepancy will be discussed further
in sections XXX and XXX.

\subsection{Solar Neutrinos}
Nuclear reactions in the core of the sun provide energy to maintain a equilibrium
between gravitational forces and XXX forces.
There exists two seperate chains of nuclear reactions that are present in typical
stellar conditions, the $pp$-chain and the CNO-cycle. Figure XXX shows these
two reaction chains.
For the Sun the $pp$ chain
provides 99\% of the generated nuclear energy, and the CNO-cycle provides the remaining
1\%. For stars significantly more massive than the Sun, the CNO-cycle is the
main energy generating mechanism.

Within the $pp$-chain there are five process that produce neutrinos.
Since the Q-value the processes in the $pp$ chain are all well below the rest mass
of a muon or tau, the only charged lepton generated is electrons. And so from
lepton flavor conservation only electron flavor neutrinos are generated.
These neutrinos are produced with an energy spectrum shown in Fig. XXX.

The $hep$ and $\ce{^{8}B}$ reactions produce neutrinos with the highest
energies. Since the $hep$ reaction branching ratio is so low the flux
of $hep$ neutrinos is also very low compared to that of $\ce{^{8}B}$ neutrinos;
the $hep$ flux is expected to be XXX\% of the $\ce{^{8}B}$ flux.
So for water-cherenkov detectors that have a typical threshold of a few MeV, $\ce{^{8}B}$ neutrinos
are the primary source of detectable solar neutrinos.

The uncertainty on the predicted $\ce{^{8}B}$ flux is relatively large, this comes mostly
from the uncertainty on the cross-sections and how those cross-sections change with
temperature, and uncertainties on the temperature profile within the core of the sun.
And since the $\ce{^{8}B}$ reaction has five preceding reactions the uncertainty on
 those reactions are part of the uncertainty on the $\ce{^{8}B}$ flux.

The uncertainty on the $pp$ and $pep$ neutrinos is much lower for two reasons. First, because
they are at early stage of the reaction chain, so their reaction rate is not dependent on any
other preceding interaction. The $pp$ reaction is also the main energy generating mechanism
for the Sun, so measurements of the total solar luminosity provide strict constraints on the
$pp$ flux as well.

Neutrinos created in the solar core can experience significant mixing effects from local
electron density.
One of the most interesting aspects to neutrino mixing within the Sun is the MSW-effect,
at a specific electron density a resonance occurs and neutrinos are maximally mixed.
The condition for an MSW-resonance between any two matter states is given by
\begin{equation}
    N_{e} = \frac{\Delta m^{2} \cos2\theta}{2\sqrt{2}EG_{F}}\text{.}
\end{equation}
This condition is met for a 10\,MeV at a solar radius of XXX, for the mixing parameters
given in XXX %cite pdg probably.
For neutrinos below XXX\,MeV this condition is not met at any point within the sun,
and so those neutrinos do not experience the MSW resonance.
Once a neutrino created in the core of the sun has travelled past a solar radius of $\approx$XXX
the solar electron density has dropped far enough that matter effects are no longer significant
and neutrinos are effectively travelling through vacuum. Once in the vacuum mixing dominated region

The effect neutrino mixing has on the neutrino flux is typically summarized by the
