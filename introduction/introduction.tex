
%: ----------------------- introduction file header -----------------------
\chapter{Introduction}

% the code below specifies where the figures are stored
\ifpdf
    \graphicspath{{introduction/figures/PNG/}{introduction/figures/PDF/}{introduction/figures/}}
\else
    \graphicspath{{introduction/figures/EPS/}{introduction/figures/}}
\fi

% ----------------------------------------------------------------------
% ----------------------- introduction content -------------------------
% ----------------------------------------------------------------------

\section{Neutrinos}
Neutrinos were first hyptothesized by Wolfgang Pauli in 1930.
The motivation for the proposal the apparent violation of energy
conservation in $\beta$ decay \citep{pauli_letter}.
Several years after Pauli's somewhat speculative proposal Enrico Fermi offered
a more thorough model of beta decay that conserved energy using the neutrino
\citep{fermi_beta_decay}.
Fermi's model predicted such a small cross-section for the neutrino that some
doubted it would ever be observed \citep{bethe_impossible_to_observe}.
However, roughly two decades after its initial proposal, Frederick Reines \&
Clyde Cowan performed an experiment that involved bombarding a tank of cadmium
doped water with anti-neutrinos from nuclear reactor.
Doing this they were able to observe the rate and energy of inverse $\beta$
decays that ocurred.
The results were consitent with Fermi's model of $\beta$ decay and were
considered a confirmation of the neutrino's existence.

\subsection{Neutrino Flavor}
The first experimental evidence for neutrino flavor came in 1962 from an
experiment \citep{lederman_muon_flavor} that observed the interactions of
neutrinos that came from muon decay, and the interactions of neutrinos emitted
from beta decay.
The experiment observed that neutrinos from came from muon decay would produce
muons upon interacting in a detector.
And neutrinos produced from $\beta$ decay would create electrons in the
detector.
This lead to the conclusion that there are two different varieties of neutrino,
the $\nu_e$ and the $\nu_{\mu}$, and the idea that lepton flavor is conserved.
The third lepton generation, the $\tau$ and the $\nu_{\tau}$ was discovered 13
years later in 1975 \citep{tau_discovery}.
% It's worth noting that lepton flavor conservation was not predicted or required
% by the standard model, for that reason it's known as an 'accidental symmetry'.
% ^^^^Not true??? Lepton number is an accidental symmetry, idk about lepton
% flavor Also the definition of accidental symmetry has to do with term in the
% lagrangian being to high dimensional (according to wikipedia) not what I
% said....though they might be the same somehow

\subsection{Neutrino Oscillations}
Neutrino oscillation is the idea that a neutrino created as a $\nu_e$ can be
detected and a $\nu_\mu$ or a $\nu_\tau$.
Oscillation occurs because the eigenstates of the neutrino's Hamiltonian in a
vaccume is different from the weak interaction eigenstates.
More succinctly put, the neutrinos mass states are not the same as
the weak states.
The mathematical description of this is as follows.
Starting with the claim that weak states and mass states are related to
each other via a rotation matrix.
\begin{equation}
    \ket{\nu_{i}} = U_{i\ell}\ket{\nu_\ell}
\end{equation}
Where $\ket{\nu_\ell}$ represents the neutrino weak states, $\ket{nu_i}$
represents the mass states, and $U_{i\ell}$ describes the mixing of these
states.
In the simplest case where the weak states and the mass states are the same
$U_{i\ell}$ would just be the identity matrix.
Additionally $U_{i\ell}$ must be unitary to conserve probability.
% TODO^^^Improve that, maybe not give a full proof but at least better wording
The mass states are defined such that

\begin{equation}
    H \ket{\nu_i} = m_i \ket{\nu_i}
\end{equation}

\subsection{Solar Neutrinos}
\subsubsection{The MSW Effect}
\subsection{Neutrino Experiments}
\subsubsection{Solar Experiments}
\subsubsection{Terrestrial Experiments}


