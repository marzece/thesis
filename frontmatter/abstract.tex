
% Thesis Abstract -----------------------------------------------------

\begin{abstracts}
    This thesis provides a description of two physics results related
    to solar neutrinos.
    The first is the development of a novel model for neutrino mixing
    that supposes there may exists a potential that couples to neutrino
    flavor and is only significant in areas of near zero matter density.
    A simulation of this model is described. Special consideration
    is given to a previously observed tension in the measured value
    for $\Delta m^{2}_{21}$ by the KamLAND reactor experiment and
    a combined solar neutrino experiment measurement.
    It's determined that this model, as developed here, does not
    provide a significant reduction in this tension.
    Best fit parameters for this model is preferred over the
    null hypothesis at XXX$\sigma$.

    The second topic is
    a measurement of the $\ce{^{8}B}$ solar neutrino flux using a
    69.2\,kt-day dataset acquired with the
    SNO+ detector during its water
    commissioning phase.
    At energies above 6\,MeV the dataset is an extremely
    pure sample of solar neutrino elastic scattering events, owing primarily to
    the detector's deep location, allowing
    an accurate measurement with relatively little exposure. In that energy
    region the best fit background rate is
    $0.25^{+0.09}_{-0.07}$\,events/kt-day, significantly
    lower than the measured solar neutrino event rate in that energy range,
    which is $1.03^{+0.13}_{-0.12}$\,events/kt-day. Also using data below this threshold, down
    to 5\,MeV, fits of the solar neutrino event direction yielded an observed
    flux of
    $2.53^{+0.31}_{-0.28}$(stat.)$^{+0.13}_{-0.10}$(syst.)$\times10^6$\,cm$^{-2}$s$^{-1}$,
    assuming no neutrino oscillations. This rate is
    consistent with matter enhanced neutrino oscillations and measurements from
    other experiments.
\end{abstracts}

% ----------------------------------------------------------------------
