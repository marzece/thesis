
% Thesis Abstract -----------------------------------------------------

\begin{abstracts}
    This thesis provides a description of two physics results related
    to solar neutrinos.
    The first topic is
    a measurement of the $\ce{^{8}B}$ solar neutrino flux using a
    69.2\,kt-day dataset acquired with the
    SNO+ detector during its water
    commissioning phase.
    At energies above 6\,MeV the dataset is an extremely
    pure sample of solar neutrino elastic scattering events, owing primarily to
    the detector's deep location, allowing
    an accurate measurement with relatively little exposure. In that energy
    region the best fit background rate is
    $0.25^{+0.09}_{-0.07}$\,events/kt-day, significantly
    lower than the measured solar neutrino event rate in that energy range,
    which is $1.03^{+0.13}_{-0.12}$\,events/kt-day. Also using data below this threshold, down
    to 5\,MeV, fits of the solar neutrino event direction yielded an observed
    flux of
    $2.53^{+0.31}_{-0.28}$(stat.)$^{+0.13}_{-0.10}$(syst.)$\times10^6$\,cm$^{-2}$s$^{-1}$,
    assuming no neutrino oscillations. This rate is
    consistent with matter enhanced neutrino oscillations and measurements from
    other experiments.

    The second is the development of a novel model for neutrino mixing
    that considers a potential that couples to neutrino
    flavor and is only significant in areas of near zero matter density.
    A neutrino mixing simulation of this model was developed and is described.
    Special consideration
    is given to a previously observed tension in the measured value
    for $\Delta m^{2}_{21}$ by the KamLAND reactor experiment and
    a combined solar neutrino experiment measurement.
    Performing a global fit to solar neutrino and reactor neutrino
    data it's determined that this new neutrino mixing model reduces the tension
    between solar neutrino measurements and the KamLAND measurement
    from $Delta chi^{2}=4.2$ to $\Delta \chi^{2}=.85$ with
    two new degrees of freedom.
    Providing a modest preference for this new model over standard neutrino mixing.
    Finally, future improvements and generalizations to this result are discussed.
\end{abstracts}

% ----------------------------------------------------------------------
