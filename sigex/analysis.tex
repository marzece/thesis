\chapter{Analysis}
\label{sec:analysis}
\section{Signal Extraction}
Once an MC simulated dataset and a detector dataset are selected, the analysis
of those events is performed.
The first steps of the analysis is to bin all events
in a two-dimensional histogram of reconstructed energy and $\cos\theta_{\mathrm{sun}}$.
Events are distributed across $N_{\theta}$ equal width bins in $\cos\theta_{\mathrm{sun}}$ from
\numrange{-1}{1} and
$N_{E}$ bins in energy from \numrange{5.0}{15.0}\,MeV.
For this analysis $N_{\theta}$ is 40 and $N_{E}$ is 6.
From \numrange{5.0}{10.0}\,MeV $5$ bins of width $1$\,MeV are used and
a single bin from \numrange{10.0}{15.0}\,MeV is used.
Simulated and detected events are placed into separate histograms.

Simulated $\nu_{e}$ and $\nu_{\mu}$ events are histogrammed separately,
but given different weights in their respective histogram according to the expected survival probability
for each event.
The weight for a $\nu_{e}$ event with neutrino energy $E_{\nu}$ is given by,
\begin{equation}
    w_{\mathrm{e}} = P_{\mathrm{ee}}(E_{\nu})\text{,}
\end{equation}
and the weight for a $\nu_{\mu}$ event is given by,
\begin{equation}
    w_{\mathrm{\mu}}= P_{\mathrm{e\mu}} = 1 - P_{\mathrm{ee}}(E_{\nu})\text{.}
\end{equation}

As mentioned in Sec~\ref{sec:simulation} the effective flux of the solar
simulations are much larger to than the expected flux of the detected dataset;
this scaled flux is the same as if the simulated dataset had a much higher livetime
but a standard flux value.
So the simulated histograms are
scaled to by $\frac{t_{live}}{t_{sim}}$ to make the effective simulated
livetime match the livetime of the detector dataset.
An additional scaling is done to the simulated histograms to account
for the data cleaning, this is needed because data cleaning
is not applied to simulated events.
The data cleaning rejection rate was determined to be $1.2\%$, so the
simulated histograms are scaled by a factor of $0.988$.

Once these scales are applied the $\nu_{e}$ and $\nu_{\mu}$ histograms
respectively  represent the expected distribution and event rate for the charged current
and neutral current interactions in the dataset for the nominal $\ce{^{8}B}$
solar neutrino flux used in simulation.
The $\nu_{e}$ and $\nu_{\mu}$ histograms are then combined by
simply adding their bin contents together to get the expected
mixed flavor elastic scattering interaction rate as a function
of $T_{\mathrm{e}}$ and $\cos\theta_{\mathrm{sun}}$.
Applying additional scaling to this combined histogram is done to represent
a hypothesized scaling to the overall solar neutrino flux.
So the rate of solar neutrino events is given by
\begin{equation}
    R_{\nu}(\cos\theta_{\mathrm{sun}}, T_{\mathrm{e}}) =
    S\phi(R)
    \label{eqn:solar_rate}
\end{equation}

Equation~\eqref{eqn:solar_rate} is modified slightly to include a parameter
related to the detector angular resolution, $\delta_{\theta}$,
\begin{equation}
R_{\nu}(\cos\theta_{\mathrm{sun}}, T_{\mathrm{e}}) \rightarrow
    R_{\nu}(\cos\theta_{\mathrm{sun}}, T_{\mathrm{e}}, \delta_{\theta})\text{.}
\end{equation}
This modification is discussed more in Sec~\ref{sec:angular_systematics}.

No simulation or measurements were done for the expected backgrounds for this analysis,
so a simple background model is adopted.
It is assumed that the direction of any background event will be uncorrelated
with the position of the sun, this is what makes $\cos\theta_{\mathrm{sun}}$ such a
useful variable in this analysis.
The distribution of background events is given by
\begin{equation}
    R_{\mathrm{B}}(\cos\theta_{\mathrm{sun}}, T_{\mathrm{e}}) =
    R_{\mathrm{B}}(T_{\mathrm{e}}) = \frac{1}{N_\theta}n_{\mathrm{B}}(T_{\mathrm{e}})
\end{equation}
Where $n_{\mathrm{b}}(T_{\mathrm{e}})$ is number of background events in the
histogram energy bins corresponding to the energy $T_{\mathrm{e}}$.
The number of background events in each energy bin are not known \textit{a~priori}
and are treated as a nuisance parameter in the remainder of the analysis.

The total expected events in each bin can be expressed by
\begin{equation}
    R(\cos\theta_{\mathrm{sun}}\text{, }T_{\mathrm{e}}) =
    R_{\mathrm{B}}(T_{\mathrm{e}}) + R_{\nu}(\cos\theta_{\mathrm{sun}}\text{, }T_{\mathrm{e}}\text{, }\delta_{\theta})\text{.}
    \label{eqn:event_rate}
\end{equation}

The unknown parameters of this rate are the 6 background rates and the solar
rate and $\delta_{\theta}$.
A fit to data is performed for equation~\eqref{eqn:event_rate} to extract
those parameters.
Goodness-of-fit is evaluated using a likelihood given by
\begin{align}
    \mathcal{L}(S&, \bold{B}, \delta_{\theta} | \bold{n}, \mu_{\theta}, \sigma_{\theta}) = \nonumber\\
    &\mathcal{N}(\delta_{\theta}, \mu_{\theta}, \sigma_\theta)
    \prod_{j=0}^{N_E}\prod^{N_{\theta}}_{i=0} \text{Pois}\left(n_{ij},\, B_{j} + S\,p_{ij}(\delta_{\theta})\right)\text{.}
    \label{eq:ll}
\end{align}
The parameters $\mu_{\theta}$ and $\sigma_{\theta}$ are respectively the best
fit and the constraint on $\delta_{\theta}$ from the $\ce{^{16}N}$ source
analysis.
$\text{Pois}\left(k, \lambda \right)$ is the value of the Poisson distribution
at the value $k$ for a rate parameter $\lambda$.

Equation~\eqref{eq:ll} can be modified to fit for the solar rate in individual
energy bins, rather than constraining the solar rate to be the same across
all energy bins. This is done by replacing the product over energy bins,
$\prod_{j=0}^{N_E}$, with the selection of $j=0\text{, }1\text{, }$\textit{etc.}
Fitting for a solar rate in each bin allows for a spectral measurement of the
solar neutrino flux, as opposed to an integrated flux measurement.
