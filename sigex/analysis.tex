\section{Analysis}
Once a MC simulated dataset and a detector dataset are selected, the analysis
of those events is performed.
The first steps of the analysis is to bin all events
in a two-dimensional histogram of reconstructed energy and $\cos\theta_{sun}$.
Events are distributed across $N_{\theta}$ equal width bins in $\cos\theta_{sun}$ from
\numrange{-1}{1}and
$N_{E}$ bins in energy from \numrange{5.0}{15.0}\,MeV.
For this analysis $N_{\theta}$ is 40 and $N_{E}$ is 6.
From \numrange{5.0}{10.0}\,MeV $5$ bins of width $1$\,MeV are used and
a single bin from \numrange{10.0}{15.0}\,MeV is used.
Simulated and detected events are placed into seperate histograms.

Simulated $\nu_{e}$ and $\nu_{\mu}$ events are histogrammed separately,
but given different weights in their respective histogram according to the expected survival probability
for each event.
The weight for a $\nu_{e}$ event with neutrino energy $E_{\nu}$ is given by,
\begin{equation}
    w_{\mathrm{e}} = P_{\mathrm{ee}}(E_{\nu})\text{,}
\end{equation}
and the weight for a $\nu_{\mu}$ event is given by,
\begin{equation}
    w_{\mathrm{\mu}} = 1 - P_{\mathrm{ee}}(E_{\nu})\text{.}
\end{equation}

As mentioned in Sec~\ref{sec:solar_simulation}%XXX
the effective livetime of the solar simulations are much larger to than
the livetime of the detected dataset; the simulated histograms are
scaled to by $\frac{t_{live}}{t_{sim}}$ to make the effective simulated
livetime match the livetime of the detector dataset.
An additional scaling is done to the simulated histograms to account
for the data cleaning sacrifice, this is needed because data cleaning
is not applied to simulate events.
The data cleaning sacrifice was determined to be $1.2\%$, so the
simulated histograms are scaled by a factor of $0.988$.

Once these scalings are applied the $\nu_{e}$ and $\nu_{\mu}$ histograms
respectively  represent the expected distribution and event rate for the charged current
and neutral current interactions in the dataset for the nominal $\ce{^{8}B}$
solar neutrino flux used in simulation.
The $\nu_{e}$ and $\nu_{\mu}$ are then combined by
simply adding their bin contents together to get the expected
flavor independent elastic scattering interaction rate as a function
of $T_{\mathrm{e}}$ and $\cos\theta_{sun}$.
Applying additional scaling to this combined histogram is done to represent
a hypothesized scaling to the overall solar neutrino flux.

Since no simulation was done for the expected backgrounds for this analysis
a simple background model is adopted.
It is assumed that the direction of any background event will be uncorrelated
with the position of the sun, this is what makes $\cos\theta_{sun}$ such a
useful variable in this analysis.
The distribution of background events is given by
\begin{equation}
    R_{\mathrm{background}}(\cos\theta_{sun}, T_{e}) = \frac{1}{N_\theta}N(T_{e})
\end{equation}

The solar neutrino flux present in the dataset is determined by first
rejecting events from the dataset that are unlikely to be solar neutrino events.
Events that pass all cuts are two-dimensionally histogrammed in kinetic energy $T_{e}$ and
$\cos\theta_{sun}$.
The range of energies considered is $5\text{\,MeV} > T_{e} < 15\text{\,Mev}$.
That range of energies was chosen to minimize contamination from radioactive
backgrounds and atmospheric neutrino interactions.
Between energies \numrange{5}{10}\,MeV the histogram bin width
is 1\,MeV, above 10\,MeV a single bin is used.
Forty bins of equal width are used for histogramming events in $\cos\theta_{sun}$.

Simulated events are used to estimate the expected distribution of events in
$\cos\theta_{sun}$ and electron recoil energy $T_{e}$.
Simulated events are treated the same was as detector events, the same
cuts are applied to their reconstructed quantities.

A position dependent correction is applied to the reconstructed energy
of both MC simulated events and detector events.

The simulated events are histogrammed to estimate the underlying PDFs
of observable reconstructed energy and event direction. Cut's are placed
on each event to
