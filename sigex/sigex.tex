
%: ----------------------- sigex file header -----------------------
\chapter{Signal Extraction}

% the code below specifies where the figures are stored
\ifpdf
    \graphicspath{{sigex/figures/PNG/}{sigex/figures/PDF/}{sigex/figures/}}
\else
    \graphicspath{{sigex/figures/EPS/}{sigex/figures/}}
\fi

% ----------------------------------------------------------------------
% ----------------------- sigex content -------------------------
% ----------------------------------------------------------------------

\section{Signal Extraction}
Solar neutrino events are monte-carlo simulated using ``RAT'', a Geant-4 based
simulation and analysis toolkit.
RAT simulates all effects after the initial interaction, including photon
propagation and detection, and particle scattering.
Beyond the photon and physics simulation RAT also simluates the SNO+ DAQ and trigger electronics,
allowing the effects of digitization and electronics noise to be simulated.

A solar neutrino production rate is an input to the simulation.
A cross-section model take from \cite{bahcall} and \cite{lookitup}
is used to estimate the elastic scattering interaction rate.
RAT provides an accurate model of the detector response for each interaction.

Solar neutrino events are simulated on run-by-run basis with a fixed average rate
of interactions.
Each run's simulation is matched to the detector trigger and
daq settings for that run. To ensure adequate monte-carlo statistics the
rate of solar $\nu_{e}$ and $\nu_{\mu\text{,\,}tau}$ interactions is artificially
enhanced by a factor of XXX and XXX; the enhanced rate is later removed as a correction
to the normalization of the PDFs created from the monte-carlo simulation. 

$\theta_{sun}$ is defined by 
\begin{equation}
\cos\theta_{sun} = \vec{d}\cdot\vec{d_{sun}}\text{, }
\end{equation}
where $\vec{d_{e}}$ represents the reconstructed direction of an event and
$\vec{d_{sun}}$ is the direction vector pointing from the center of the sun to
the reconstructed position of the event.
$\vec{d_{sun}}$ estimates the direction the neutrino was travelling when it interacted
within the detector, $\vec{d_{\nu}}$; it is assumed the neutrino travelled directly 
from the center of the sun without scattering off anything while it travelled.
This is a good assumption because the neutrino cross-section is small that it's very
unlikely the neutrino will interact with anything before interacting in the detector.
Assuming the neutrino comes from the center of the sun is a poor assumption for an individual
neutrino, but averaged over many neutrinos it is a good assumption.
Additionally, correcting for the radius the neutrino is produced at would only adjust
the direction by at most $0.1\deg$.
Figure XXX shows the angle between $d_{sun}$ and $d_{\nu}$ for simulated solar neutrino
events.

Figure XXX shows why $\theta_{sun}$ is a useful variable for a solar neutrino analysis.
By comparing the rates of events with different values for $\theta_{sun}$ one can
extract a background rate and a solar rate.

\subsection{Reconstruction}
A series of reconstruction algorithms are ran over all events that pass data cleaning.
These algorithms estimate the position, time, direction, and energy of the event.
All events are reconstructed under the hypothesis that the PMT hits are from cherekov radiation
produced by a single electron.
The direction ($\vec{d}$), time ($\t_{0}$), and position ($\vec{p}$) are determined simultaneosly by performing a likelihood
fit to the time and position of PMT hits.
The algorithm evaluates the likelihood of a hypothesized event position and time by
calculating the time residual for each hit PMT,
\begin{equation}
\t_{res} = t_{PMT} - t_{transit} - t_{0}
\end{equation}
and using a PDF for $\t_{res}$ determined from simulation.
The algorithm uses only the events in the prompt time window to ensure only light
that travelled directly from the event origin is used.

\subsection{Calibration}


Since the cross-section for neutrino elastic scattering is highly peaked in the forward
direction one expects the $\vec{d_{e}}


of a vector pointing from the center of the sun at the time of the event, to the reconstructed position of the event.


Simulated events are used to estimate the distribution of events in $\cos\theta_{sun}$ and
electron recoil energy $T_{e}$. 

The simlated events are histogrammed to estimate the underlying PDFs
of observable reconstructed energy and event direction. Cut's are placed
on each event to 

\subsection{Data Cleaning}
\subsubsection{CAEN Cut}

