\section{Calibration}
The accuracy of simulated events is evaluated with data taken while a
radioactive source was deployed within the detector volume.
For this analysis was an $\ce{^{16}N}$ source was used.

The $\ce{^{16}N}$ source was developed by SNO, it uses a commercial
deuterium and tritium generator (DT-generator) to produce gaseous $\ce{^{16}N}$.
The gas is pumped into the deployed source where it can undergo $\beta$-decay
to an exciteed state of $\ce{^{16}O}$, the $\ce{^{16}O}$ will then de-excite
and typically emit a $XXX$\,MeV gamma particle. Higher energy gammas are emitted
at a lower rate, the branching ratios for the de-excitation gammas are shown in
figure XXX.

A small block of plastic scintillator, observed by a PMT, is embedded within the
source cannister. That PMT can detect the $\beta$ from the initial
$\ce{^{16}N} \rightarrow \ce{^{16}O}$ decay. That PMT signal is used as a
tag in the detector DAQ to identify events from the deployed source.

The source position within the AV was varied in a 3-dimensional
scan.
A 1-dimensional scan was done along the z-axis outside the AV volume,
but inside the PSUP, as well.
Scanning many positions allowed for a position dependent evaluation of
systematics.

\subsection{Energy Calibration}
The detector resolution $\sigma_{E}$ and relative energy scale $\delta_{E}$ are determined
from the $\ce{^{16}N}$ energy spectrum.
The energy spectrum is modeled by $P(T_\mathrm{e})$, the energy spectrum in
electron equivalent kinetic energy, and is given by $P_{\mathrm{source}}(T_{\mathrm{e}})$
convolved with a normalized Gaussian distribution,
\begin{equation}
    P(T_\mathrm{e}) = N \int P_\mathrm{source}(T_{e})\frac{1}{\sqrt{2\pi}\sigma_{E}}e^-{\frac{\left((1+\delta_E)T_\mathrm{e}-T^{\prime}_{e}\right)^{2}}{2\sigma^{2}_{E}}}dT^{\prime}_{e}\text{.}%-\Delta E
\label{eq:convolution}
\end{equation}
$P_\mathrm{source}(T_{\mathrm{e}})$ represents the distribution of deposited energy in
the detector from the $\ce{^{16}N}$ source, in electron equivalent energy.
Since the $\ce{^{16}N}$ emits gammas into the detector the mapping between
gamma energy and electron equivalent energy is done by finding the electron
energy that can produce the same number of Cherenkov photons
as each gamma; this is not a one-to-one mapping because the same electron or gamma
energy will not always produce the same number of photons.
The mapping is determined from simulation and is shown in
figure XXX.
The gamma to electron energy mapping is then applied to the simulated $\ce{^{16}N}$
gamma energy spectrum to determine $P_\mathrm{source}(T_{\mathrm{e}})$.

The values for $\sigma_{E}$ and $\delta_{E}$ are extracted from~\eqref{eq:convolution}
by performing a fit to the reconstructed $\ce{^{16}N}$ energy spectrum.
The fit is done to both simulated $\ce{^{16}N}$ data and to detector data,
each determining their own values for $\sigma_{E}$ and $\delta_{E}$. It's worth
noting that $\sigma_{E}$ represents only the resolution provided by detector
effects, resolution from effects such as photon statistics are accounted
for in $P_\mathrm{source}(T_{\mathrm{e}})$

Values for $\sigma_{E}$ and $\delta_{E}$ are extracted for data taken, or simulated, with
the $\ce{^{16}N}$ source at many position, allowing for a position dependent
determination of the energy scale and resolution.
Fitting to both simulated and to detected data allows for a correction to
be created that can make the two datasets match better, however the data
used to create the correction cannot then be used to determine systematics.
So the $\ce{^{16}N}$ data was split into two datasets, one for determining
what correction should be applied to simulation, the other for extracting systematics
after the correction is applied.

The data for the correction is futher divided into position bins in $z$ and
$\rho$, where $\rho = \sqrt{x^{2} + y^{2}}$. The choice of binning is motivated
by the symmetry of the detector, the detector is very symmetric for an interchange
of $x$ and $y$ or $x,y \rightarrow -x,-y$.
There exists, however, signficant assymetries along the $z$ axis
from the detector neck and from the rope-net along the top of the AV.
The data is divided into 4-bins along the $\rho$ direction each $200$\,cm long
and bins of $57$\,cm height along the $z$ axis. The number of bins along the
z-axis varies for each slice in $\rho$ because data was primarily taken within
the AV.\@
Figure XXX shows the fits for $\delta_{E}$ and $\sigma_{E}$ in each bin for
both simulated and detected events.

Variations in $\delta_{E}$ along $z$ and $\rho$ were modeled by a polynomial given by,
\begin{equation}
    \delta_{E}(\rho^{2}, z) = A + \left[(1+B\rho^{2})(1+Cz+Dz^{2}+Ez^{3}) - 1\right]\text{.}
    \label{eqn:escale_position_dependence}
\end{equation}
Values for $A$, $B$, $C$, $D$, and $E$ are extracted from a fit to the observed
spatial variation of $\delta_{E}$ for simulation and data and are given in
table~\ref{tbl:n16_position_escale}. The reconstructed energies of simulated and detected 
events are then corrected according to~\eqref{eqn:escale_position_dependence} by
their respective best fit values.
The energy resolution is evaluated as a function of position but no correction
is determined from it.
\begin{table}
    \centering
\begin{tabular}{|c | c | c | c |c|c|}
    \hline
    & A&B&C&D&E\\
    \hline
    Data& 2.53e-2& 1.48-e9 & -5.44e-6 & 2.14e-9 & 6.49e-13\\
    Simulation& 3.33e-2& 9.48e-10& 3.77e-6& 4.46e-10& 1.43e-13\\
    \hline
\end{tabular}
    \caption{Best fit values for~\eqref{eqn:escale_position_dependence} for
    simulated and detected data, determined using units of mm for $z$ and $\rho$.}
    \label{tbl:n16_position_escale}
\end{table}


After the correction is applied to remaining half of the calibration dataset
$\sigma_{E}$ and $\delta_{E}$ are determined once more as a function of position.
The bin-by-bin differences in $\sigma_{E}$ and $\delta_{E}$ between
simulated and detected data are taken as the systematic uncertainty for those
parameters, with additional fit uncertainties added in quadrature.
Averaging the bin-by-bin systematic uncertainty over the detector volume relevant
for the solar analysis yields a $2.5\%$ uncertainty on $\delta_{E}$ and an
$11\%$ uncertainty on $\sigma_{E}$.

\subsection{Position Calibration}
Similar to the energy calibration, the position reconstruction is evaluated
using $\ce{^{16}N}$ data and simulation.
The difference between the source position and the reconstructed
position of each event is determined and histogrammed.
A fit to that distribution is performed using a model of a Gaussian distribution
with exponential tails convolved with a distribution for the first gamma interaction
distance. The equation for this is given by,
\begin{equation}
    P(x)  = A \cdot \bigg[ \bigg(\frac{1 - \alpha}{\sqrt{2\pi}\sigma}e^{- \frac{(x-\mu)^2}{2\sigma^2}} + \frac{\alpha }{2 \tau}e^{-\frac{|x-\mu|}{\tau}}\bigg) \circledast P_{\gamma}(x) \bigg]\text{.}
    \label{eqn:n16_position_model}
\end{equation}
Where $\mu$ and $\sigma$ are respectively the center and width of the Gaussian,
$\tau$ represents the decay rate for the exponential tails, and $\alpha$ represents
the relative strength of the exponential~vs.~the Gaussian;
$P_{\gamma}(x)$ is distribution of distance travelled by an $\ce{^{16}N}$
gamma before it's first interaction, it is determined from a separate MC
simulation.
Finally $A$ is an overall normalization to account for the number of events
included in the distribution.
The Gaussian and exponential portion of
\eqref{eqn:n16_position_model} represents the spread introduced by the detector
and position reconstruction, the $P_\gamma$ term represents the intrinsic spread
in interaction positions from the source itself.
Figure XXX shows an example of this distribution and fit for a central $\ce{^{16}N}$
dataset.

With this scheme three types of position uncertainties are considered, a shift
uncertainty, a resolution uncertainty, and a scale uncertainty.
Here a position shift is the value for $\mu$ in equation~\eqref{eqn:n16_position_model}
averaged over the entire detector volume, $\langle \mu \rangle$;
the position shift systematic then is the difference in $\langle \mu \rangle$
from MC simulation and as determined by detector data.
Rather than averaging over all source positions $\langle \mu \rangle$
is determined averaging over scans along the $x$, $y$ and $z$ axis and
so a position shift for each axial direction is determined.
Only source positions along each axis are used to avoid possible correlations
in each direction's position shift. The resulting systematic uncertainties along
each axial direction are given in table~\ref{tbl:position_shift_systs}.
\begin{table}
    \centering
    \begin{tabular}{|c|c|}
            \hline
            &$\langle \mu \rangle$ Systematic Uncertainty (mm)\\
            \hline
            x&+16.4, -18.2\\
            y&+22.3, -19.2\\
            z&+38.4, -16.7\\
            \hline
    \end{tabular}
    \caption{Position shift systematic uncertainties}
    \label{tbl:position_shift_systs}
\end{table}

The position resolution systematics is evaluated in a similar way as the position
shift systmatic, comparing values for
$\sigma^{2}$ in equation~\eqref{eqn:n16_position_model} instead of $\mu$,
but otherwise following the same procedure.
Table XXX gives the extracted position resolution systematics uncertainties
in $\text{mm}$
\begin{table}
    \centering
    \begin{tabular}{|c|c|}
            \hline
            &$\langle \sigma \rangle$ Systematic Uncertainty (mm)\\
            \hline
            x&104.0\\
            y&98.2\\
            z&106.2\\
            \hline
    \end{tabular}
    \caption{Position resolution systematic uncertainties}
    \label{tbl:position_resolution_systs}
\end{table}


For the solar analysis the most important position systematic is the vertex
scale systematic.

\subsection{Direction Calibration}
