\section{Calibration}
The accuracy of simulated events is evaluated with data taken while a
radioactive source was deployed within the detector volume.
For this analysis was an $\ce{^{16}N}$ source was used.

The $\ce{^{16}N}$ source was developed by $SNO$, it uses a commercial
deuterium and tritium generator (DT-generator) to produce gaseous $\ce{^{16}N}$.
The gas is pumped into the deployed source where it can undergo $\beta$-decay
to an exciteed state of $\ce{^{16}O}$, the $\ce{^{16}O}$ will then de-excite
and typically emit a $XXX$\,MeV gamma particle. Higher energy gammas are emitted
at a lower rate, the branching ratios for the de-excitation gammas are shown in
figure XXX.

A small block of plastic scintillator, observed by a PMT, is embedded within the
source cannister. That PMT can detect the $\beta$ from the initial
$\ce{^{16}N} \rightarrow \ce{^{16}O}$ decay. That PMT signal is used as a
tag in the detector DAQ to identify events from the deployed source.

The source position within the AV was varied in a 3-dimensional
scan.
A 1-dimensional scan was done along the z-axis outside the AV volume,
but inside the PSUP, as well.
Scanning many positions allowed for a position dependent evaluation of
systematics.

\subsection{Energy Calibration}

The relative energy scale $\delta_E$ and detector energy resolution $\sigma$
are determined by fitting the reconstructed energy spectrum $P(T_\mathrm{e})$
with the predicted apparent energy spectrum $P_\mathrm{source}(T_e)$ convolved
with a Gaussian~\cite{Dunford:2006qb}:
\begin{equation}
  P(T_\mathrm{e}) = N \int P_\mathrm{source}(T_e)\frac{1}{\sqrt{2\pi}\sigma}e^-{\frac{((1+\delta_E)T_\mathrm{e}-T\prime_{e})^2}{2\sigma^2}}dT\prime_{e}.%-\Delta E
\label{eq:convolution}
\end{equation}
Here, $\sigma$ is the detector-only resolution, which excludes the contribution
from Cherenkov photo-statistics and the production of electrons from the initial
$\gamma$'s, because these are accounted for in $P_\mathrm{source}(T_e)$.  The
detector resolution is expected to be dominated by a $\sqrt{E}$ term, owing to
variations in light collection and photo-electron production among PMTs,
yielding $\sigma(E) = b\sqrt{E}$, where $b$ is to be fit.  Other terms are
expected to contribute much less to the resolution, and for robustness, it is
preferred to have only one parameter to fit.  Fortunately, the choice of terms
in the resolution expression has little impact on the fit results.  An energy
offset or bias is not considered in the fit because
%all components of the energy reconstruction are multiplicative, and
it is assumed that there is no significant background contribution to the
measured signals.  The determination of $P_\mathrm{source}(T_{e})$ is discussed in
the next section.


Figure~\ref{XXX} shows the  distribution of EnergyRSP reconstructed energy for
$\ce{^{16}}$ events, both simulated and detected.
\subsection{Direction Calibration}
\subsection{Position Calibration}
