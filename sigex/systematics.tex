\section{Systematics}
Systematics associated with event reconstruction, livetime, mixing parameters,
and trigger efficiency are considered for this analysis.
The event reconstruction systematics are uncertainties on the energy reconstruction
scale and resolution, position reconstruction resolution and scale, and the
resolution of the direction reconstruction.
The uncertainties on each of these values and how they were determined is
discussed in Sec~\ref{sec:calib}.

Each systematics is treated in the same, or a similar, way, to propagate
their effect to the flux result.
The systematics are propagated through the analysis by modifying the
relevant quantities on reconstructed monte-carlo events according to the
one-$\sigma$ uncertainty.
The PDFs that result from the modified events are used in the analysis to extract
a flux result.
The difference between the systematically adjusted flux result and the standard
result is taken to be the one-$\sigma$ systematic uncertainty.
How each variable is modified, and any deviations from this process of
propagating systematics is detailed below.
All systematics are treated as uncorrelated, that is variables are modified according to
only one systematic at a time.

\subsection{Energy Resolution}
The energy resolution uncertainty is determined primarily from the $\ce{16}N$ analysis.
The systematic uncertainty on the energy resolution was determined to be $\delta_{\sigma} = +1.8\%, -1.6\%$. %TODO
To create the energy resolution systematic's modified PDFs the reconstructed
energy of the MC simulated events is mapped to a normalized Gaussian distribution with a
mean value of the event's energy and a
variance given by
\begin{equation}
  \sigma^{2} = \sigma_{E}^{2}\left(\left(1 + \delta_{\sigma}\right)^2 - 1\right)\text{.}
  \label{eqn:systmatic_esmear}
\end{equation}
This process of mapping a single energy value to a Gaussian distribution is referred to as ``smearing''.
Here $\sigma_{E}$ is given by $\sqrt{E}$ to match the functional form used in the fit for the systematics,
Eqn~\ref{XXX}.
The idea behind this smearing is to compensate for the possibility that our monte-carlo simulation could have
a systematically smaller energy resolution than occurs in real data.
So by applying a smearing the monte-carlo energy resolution is artificially deteriorated, and the uncertainty
on the resolution is accounted for.
A similar process does not exists to account for the possibility that the monte-carlo simulation has a poorer
energy resolution than data taken from the real detector; there's no way to ``un-smear'' the reconstructed MC
event energy.
So, to account the effect of an over-estimated energy resolution the error on the result is assumed to be symmetric.
As a penalty for this assumption the larger uncertainty between the positive and negative uncertainty on the
energy resolution is used.

If the smeared event passes all cuts then each energy binned

\subsection{Energy Scale}
Systematically varied PDFs for the energy scale PDF is generated by simply modifying the reconstructed
kinetic energy of each event according to
\begin{equation}
  T^{\prime}_{e} = (1+\delta_{E})T_{e}\text{.}
\end{equation}
At all points in the analysis afterwards $T^{\prime}_{e}$ is used instead of $T_{e}$.

\subsection{Fiducial Volume}
Uncertainty on the fiducial volume comes primarily from systematics associated
with the position reconstruction.
If the position reconstruction is more likely to pull an event towards the
middle of the detector
in MC simulation than in data, it will result in an over prediction of the
number of events that will pass the FV cut.
This possibility is accounted for by shifting the reconstructed position of
simulated events according to the uncertainty, the fiducial volume cut is
applied to those shifted positions.  Shifting the events results in modified
PDFs, those PDFs are used in the fit for the solar event rate, the difference
between the best fit value extracted with the modified PDFs and the best fit
value from the standard PDFs is taken to be the systematic uncertainty.
\subsection{Angular Resolution}
The angular resolution uncertainty is treated differently from other
uncertainties because the distribution of events in $\cos\theta_{sun}$ is
directly related to the direction resolution.  To minimize the impact the
angular resolution has on the result it is used as one of the parameters in the
fit to the $\cos\theta_{sun}$ solar neutrino distribution, and constrained by
the results of the $\ce{^{16}N}$ analysis.

The angular resolution systematic is applied using the formula given in
~\ref{sec:angular_syst},
\begin{equation}
    \cos\theta^{\prime} = 1 + (\cos\theta - 1)(1+\delta_{\theta})\text{.}
    \label{eqn:angular_syst_map}
\end{equation}
Where $\theta$ is the angle between the true event direction and the reconstructed
event direction, and $delta$ is the angular resolution systematic uncertainty.
Using~\eqref{eqn:angular_syst_map} has the unfortunate downside producing
unphysical values for $\cos\theta^{\prime}$ for values of $\cos\theta$ near
$-1$. For values of $\cos\theta^{\prime}$ below $-1$ the value is instead replaced
with a random value drawn from a uniform distribution $\left[-1\text{, }1\right]1$.
The logic behind this choice is that when an event reconstructs with a direction
that's nearly $180^{\circ}$ from the correct value, then the reconstruction
has likely failed to such a degree that the reconstructed values are uncorrelated
with the true values, and so drawing from a uniform random distribution preserves
that uncorrleated nature without adding any additional bias.

Once the systematically varied value for $\cos\theta$ is determined, the new angle
needs to be transformed into a corresponding direction vector for the particle.
To do this first a vector that is normal to the plane spanned by the reconstructed
and true direction vector is found by taking the cross-product between those vectors,
\begin{equation}
    \vec{v}_{\mathrm{norm}} = \vec{d}_{\mathrm{true}} \cross \vec{d}_{\mathrm{recon}}\text{.}
\end{equation}
Then $\vec{d}_{\mathrm{recon}}$ is rotated around $\vec{v}_{\mathrm{norm}}$
such that the rotated vector $\vec{d}^{\prime}_{\mathrm{recon}}$ now has an angle
of $\theta^{\prime}$.
The direction $\vec{d}^{\prime}_{\mathrm{recon}}$ is then used in the
analysis to generate event distrubtions in $\cos\theta_{sun}$.

Following this procedure PDFs for $\cos\theta_{sun}$ are generated for many differnt
values of $\delta_{\theta}$, producing $P(\cos\theta_{sun}, \delta_{\theta})$.
The constraint on $\delta_{\theta}$ produced by the $\ce{^{16}N}$ analysis
are included in this two-dimension PDF\@.
$P(\cos\theta_{sun}, \delta_{\theta})$ is used in the remaineder of the analysis
treating $\delta_{\theta}$ as a nuisance parameter.

\subsection{Mixing Parameters}
The central values and uncertainties of the neutrino mixing parameters, $\Delta
m^{2}_{21}$, $\theta_{12}$ and $\theta_{13}$ is taken from Ref.~\citep{pdg_globalfit}.
\begin{table}
    \centering
    \begin{tabular}{c c c}
        Parameter & Value & Uncertainty\\
        $\Delta m^{2}_{21}$ & $7.37\times10^{-5} \mathrm{MeV}/\mathrm{c}^{2}$ & +0.17, -0.16\\
        $\theta_{12}$ & $33.02^{\circ}$ & 0.537, -0.455 \\
        $\theta_{13}$ & $8.41^{\circ}$ & -- \\
        $\Delta m_{31}$ & $2.5e-3\times10^-3\mathrm{MeV}/\mathrm{c}^{2}$ & -- \\
    \end{tabular}
    \caption{A summary of the mixing parameters and their uncertainties as used in 
    propagation of systematic uncertainties. Values from from Ref~\citep{pdg_globalfit}.}
\label{tbl:mixing_values}
\end{table}
The values and uncertainties for the mixing parameters
are summarized in Table~\ref{tbl:mixing_values}.
A survival probability curve is generated for each of the mixing parameters
shifted by their positive and negative one-sigma uncertainty.
These systematically adjusted PDFs are used in the analysis replacing the
standard survial probability curve to propagate the uncertainties to the
flux result.
\subsection{Trigger Efficiency}
The uncertainty on the trigger efficiecy is described in Sec~\ref{sec:trigeff}.
PDFs for $\cos\theta_{sun}$ are generated using the more pessimistic
trigger efficiecy curves measured by the laserball and TELLIE\@.
Simulated events that have an in-time nhit that is predicted by the nhit-monitor
to be 100\% efficient are de-weighted to match the laserball/TELLIE efficiecy
measurement..
The PDFs that result from the de-weighted events are used as the systematically
adjusted PDFs to account propagate the trigger efficiecy uncertainty to the
flux result.
\subsection{Livetime}
Uncertainty on the livetime comes primarily from orphaned events in the detector.
Orphaned detector events are disucssed in~Sec.~\ref{TODOXXX}.
When an orphaned event ocurrs all information about that event is lost including
the time it ocurred and\ldots%TODO! Find documentation from Aobo.

