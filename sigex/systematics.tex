\section{Systematics}
Systematics associated with event reconstruction, livetime, mixing parameters,
and trigger efficiency are considered for this analysis.
The event reconstruction systematics are uncertainties on the energy reconstruction
scale and resolution, position reconstruction resolution and scale, and the
resolution of the direction reconstruction.

These systematics are generally treated in the same, or a similar, way, to propagate
their effect to the flux result.
The uncertainties on each quantity are determined from a separate analysis, \textit{e.g.}
from an analysis $\ce{^{16}N}$ data.
Those uncertainties are propagated through the analysis by modifying the
relevant quantities on reconstructed monte-carlo events according to the
one-$\sigma$ uncertainty.
The PDFs that result from the modified events are used in the analysis to extract
a flux result.
The difference between the systematically adjusted flux result and the standard
result is taken to be the one-$\sigma$ systematic uncertainty.
How each variable is modified, and any deviations from this process of
propagating systematics is detailed below.
All systematics are treated as uncorrelated, that is variables are modified according to
only one systematic at a time.

\subsection{Energy Resolution}
The energy resolution uncertainty is determined primarily from the $\ce{16}N$ analysis.
The systematic uncertainty on the energy resolution was determined to be $\delta_{\sigma} = +1.8\%, -1.6\%$. %TODO
To create the energy resolution systematic's modified PDFs the reconstructed
energy of the MC simulated events is mapped to a normalized Gaussian distribution with a
mean value of the event's energy and a
variance given by
\begin{equation}
  \sigma^{2} = \sigma_{E}^{2}\left(\left(1 + \delta_{\sigma}\right)^2 - 1\right)\text{.}
  \label{eqn:systmatic_esmear}
\end{equation}
This process of mapping a single energy value to a Gaussian distribution is referred to as ``smearing''.
Here $\sigma_{E}$ is given by $\sqrt{E}$ to match the functional form used in the fit for the systematics,
Eqn~\ref{XXX}.
The idea behind this smearing is to compensate for the possibility that our monte-carlo simulation could have
a systematically smaller energy resolution than occurs in real data.
So by applying a smearing the monte-carlo energy resolution is artificially deteriorated, and the uncertainty
on the resolution is accounted for.
A similar process does not exists to account for the possibility that the monte-carlo simulation has a poorer
energy resolution than data taken from the real detector; there's no way to ``un-smear'' the reconstructed MC
event energy.
So, to account the effect of an over-estimated energy resolution the error on the result is assumed to be symmetric.
As a penalty for this assumption the larger uncertainty between the positive and negative uncertainty on the
energy resolution is used.

If the smeared event passes all cuts then each energy binned

\subsection{Energy Scale}
Systematically varied PDFs for the energy scale PDF is generated by simply modifying the reconstructed
kinetic energy of each event according to
\begin{equation}
  T\prime_{e} = (1+\delta_{E})T_{e}\text{.}
\end{equation}
At all points in the analysis afterwards $T\prime_{e}$ is used instead of $T_{e}$.
\subsection{Fiducial Volume}
Uncertainty on the fiducial volume comes primarily from biases in the position reconstruction.
If the position reconstruction is more likely to pull an event towards the middle of the detector
in MC simulation than in data, it will result in an over prediction of the number of events that will
pass the FV cut.
This possibility is accounted for by shifting the reconstructed position of simulated events according
to the uncertainty, the fiducial volume cut is applied to those shifted positions.
Shifting the events results in modified PDFs, those PDFs are used in the fit for the solar event rate,
the difference between the best fit value extracted with the modified PDFs and the best fit value from
the standard PDFs is taken to be the systematic uncertainty.
\subsection{Angular Resolution}
The angular resolution uncertainty is treated differently from other uncertainties because the
distribution of events in $\cos\theta_{sun}$ is directly related to the direction resolution.
 To minimize the impact the angular resolution has on the result it is used as one of the parameters
 in the fit to the $\cos\theta_{sun}$ solar neutrino distribution, and constrained by the results
 of the $\ce{^{16}N}$ analysis.

\subsection{Mixing Parameters}
\subsection{Trigger Efficiency}
\subsection{Livetime}

