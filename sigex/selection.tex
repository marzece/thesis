\section{Blindenss}
The majority of this analyses was designed with the data blinded in the
relevant energy region.
The data blinding was done primarily for a nucleon decay analysis~\cite{ND_PAPER}
that was performed using the same dataset as this analyses, but was done
for this analysis as well because both results were produced contemporaneously.

%TODO check these numbers
The scheme for blinding the data was to remove all events from the dataset that
had an nhit between $30$ and $100$, this correspond to an approximate electron energy
range of \numrange{4.0}{15.0}\,MeV.
The analyses was designed primarily using simulation and a two-week open
data period, on which no blindness restrictions were imposed.

Part way through the analysis testing and design the data was partially un-blinded
to allow for an initial look at the results.
Blinded events that reconstructed energy between $4.5$\,MeV and $15.0$\,MeV
were added into the dataset with all energy related information removed;
each event was assigned the artificial energy of $10.0$\,MeV.
With those data, plots of the distribution of event direction with respect to
the sun were created and provided a check that the results were as expected.
The energy cut used to select events for that plot was performed using
uncalibrated energy, so it has little bearing on the results of the full
analysis.

\section{Data Selection}
Events are included or removed from the dataset across three stages of selection.
First entire runs are either included or removed based upon if they meet
certain criteria for data quality.
The events within selected runs are then rejected or approved by a set
of low-level cuts that attempt to remove events caused by instrumental
backgrounds and other sources of unwanted events.
Within events that pass the low-level cuts, hits can be rejected from consideration
if they're deemed unlikely to have originated from light within the detector.
Following that analysis level cuts are applied to the reconstructed quantities
for each event.
The analysis cuts are designed to maximize the signal efficiency for dataset and minimize the
contamination from background sources.
Each of these steps are detailed below.

\subsection{Run Selection}
Runs are rejected from the dataset if they fail to meet certain criteria
for data quality and detector stability.
It's first required that all meta-info about a run be created and stored
within the RAT Database successfully.
Those tables have information about the state of the hardware and DAQ, as
well as information about the run including the run type and length.
This information is necessary for assessing if the run is capable of being
used for a physics analysis.
It's very rare for the meta-info for a run not to be generated and stored
properly, so this check has a almost no effect on the dataset.

It's first required that the run type of the run be ``PHYSICS'', this indicates
that the DAQ settings were not changed at any point in the run and no external
sources of events were present, and that the thresholds for triggering were set
at a level deemed sufficient for most physics analyses.
Following this checks are done
that require all electronics crates have high voltage on their PMTs throughout
the run.
It's also required that all channels that are at high voltage are capable of
reading out data, and that all crates are participating in the trigger sums.
A number of separate checks are performed to ensure that all necessary DAQ
processes were running well throughout the run, to ensure that the data taken
during the run was not interrupted by a lapse in the DAQ.\@

Beyond checks on the detector state and stability, a number of checks
are placed on the data taken within the run.
Most of these checks are placed on the rate of certain events within the detector.
It's required that the ESUMH and N100L trigger rate be greater than 5Hz and
that the total trigger rate be less than 7kHz.
These checks ensure that the data taken during the run triggered the detector
at a rate consistent with standard running, during which the typical trigger
rate is near 1kHz.
Similarly it's required that fewer than 15\% of all events be retrigger events
(events that fall within 420\,ns of the previous event).
At a nominal rate of 1kHz it's very unlikely for two event to be within $420$\,ns
by chance, so a high rate of retrigger events might indicate a high level
of noise in the trigger hardware. There are however events that can occur
in the detector, such as followers after a cosmic muon, that can produce
re-trigger events.
The cut threshold is designed to allow for retriggers from natural source but
still flag detector abnormalities.
These checks all ensure that the data in a run is likely to be useful for
a physics analysis, but are loose enough not to bias the dataset in a way that
might influence results.
\subsection{Data Cleaning}
There are a number of instrumental effects that can cause an event
to be recorded by the detector, these events typically have some
sort of distinguishing feature or features that set them apart
from events that originate from particle interactions within the
detector.
A number of algorithms and cuts have been designed to identify and remove
these events from the dataset.
These algorithms are said to ``clean'' the data by removing events
of instrumental origin. These definition is extended to include removing
hits within an event that are likely of instrumental origin as well.

The primary type of instrumental event that must be removed is ``flashers'' and
``shark-fin'' events.
Both of these result from charge build-up on the PMT-base causing a
spark.
For a flasher event the light from the spark escapes through the PMT face
and illuminates the PMTs on the other side of the detector.
Flashers occur at a rate of a few per minute.
A shark-fin is similar but the spark is either small enough or located
in a position such that the light does not escape the PMT.\@
In both types of events the PMT in which the spark occurs with readout
a very high-charge hit, and the channels next to it on the FEC will
have low-charge hits from electronic pickup.
For shark-fin event no other channels will be hit, except possibly by an
accidental coincidence; for a flasher hit a number of hits will
occur from the light that escaped the PMT.\@
Since the number of PMT hits that occur in a flasher event can vary significantly,
anywhere between tens of hits and hundreds, they can reconstruct
to a wide range of energies and possibly contaminate a signal region.
Many data cleaning cuts are designed to ensure that all
flasher events are identified and removed from the dataset.

Within an event hits that are deemed unlikely to have come from a photon interacting with a PMT
are removed from the analysis through hit cleaning.
For this analysis the only sort of hits that were removed were those identified
as coming from cross-talk between adjacent channels in the detector.
Hit's from cross-talk arise from stray capacitative coupling between adjacent, or nearby, channels
on a single daughter board.
Typically the noise from cross-talk will only be large enough to cause a hit
on adjacent channels if the original signal is relatively large.
The cross-talk hits will usually be especially low in charge because they're
the result of bi-polar noise, rather than a PMT signal, and the cross-talk
hit will always show up after the original hit.
These criteria are codified as a cut on any hits that show up within
 six channels from a hit that has a pedestal subtracted QHS greater
than 50\,ADC counts. Of those hits if it has a pedestal subtracted QHS between
$10$ and $-30$ and are between 9 and 25\,ns after the high charge hit,
then the hit is flagged a cross-talk hit and removed from the analysis.

Events of instrumental origin are not well modeled within our simulation,
so it is not used for evaluating the efficiencies and sacrifices of
data cleaning cuts.
Instead a data-driven approach is used that relies primarily on calibration
data from the $\ce{^{16}N}$ source, that analysis is detailed in~\ref{XXX}.%Teals data cleaning stuff
The basic approach is to use tagged $\ce{^{16}N}$ events as a source of known
non-instrumental events, and evaluate what fraction of the time those events
are identified as instrumentals, this provides an estimate of the data-cleaning
sacrifice. The results of this analysis estimated a $XXX$\% percent signal
sacrifice from data cleaning.

An estimate of the signal contamination was performed using a method developed
by SNO~\cite{NEIL_THESIS} and applied to the dataset~\cite{DC_DOCUMENT};
the number of instrumental events leaked into the signal region was estimated
to be roughly $0.5$ events over the entire dataset.
However, a contamination estimate is not an input to the solar analysis,
so that value is not used beyond a check that the instrumental background
is reduced to a acceptable level.

Each data cleaning cut is associated with a bit in a 64-bit binary value called
the data cleaning word or data cleaning mask.
The cuts that each event passes or fails is tracked by its data cleaning mask.
For the solar analyses all events are required to pass all cuts
given by the data cleaning mask \texttt{0xFB0000017FFE}.
This corresponds to the following data cleaning cuts:
    \textbf{Zero Zero Cut},
    \textbf{Crate Isotropy Cut},
    \textbf{FTS Cut},
    \textbf{Flahser Geo Cut},
    \textbf{ITC Time Spread Cut},
    \textbf{Junk Cut},
    \textbf{Muon Tag},
    \textbf{Neck Cut},
    \textbf{Owl Cut},
    \textbf{QCluster Cut},
    \textbf{QvNhit Cut},
    \textbf{QvT Cut},
    \textbf{Ring Of Fire Cut},
    \textbf{Two-Pass Muon Follower, Short},
    \textbf{Polling Cut},
    \textbf{Retrigger Cut},
    \textbf{Two-Pass Burst Cut},
    \textbf{Missed Muon Follower Cut},
    \textbf{Missing CAEN Data Cut},
    \textbf{Ped Cut},
    \textbf{Atmospheric Cut}.
Of those cuts I'll detail here those that I developed, the rest are described
in~\ref{APPENDIXXX}.

%TODO move this to appendix probably
\subsubsection{Ped Cut}
During normal detector operations there are a few trigger calibration
tests that are periodically ran.
These tests use the PEDESTAL signal to inject a certain amount of fake hits
into the detector, and events with those hits are inspected to evaluate the efficiency and
quality of the trigger response.
It's very important that these events are clearly identified and removed from the
dataset so that the fake PEDESTAL hits are not confused for a real signal.
Additionally, the trigger calibration processes usually include changing settings
related to the PEDESTAL signal on the FEC, there's reason to believe these sort
of changes can introduce noise to the front-end.
So an aggressive approach of cutting all events that are within one second of
a pedestal event is used.
This not only cuts events but introduces a dead time into the dataset,
this deadtime is subtracted from the overall livetime.

\subsubsection{Missed Muon Follower Cut}
The missed muon follower cut was a data cleaning cut used in SNO, but I adapted
it for SNO+.
In SNO it was very important to identify and cut neutrons that follow
after a cosmic muon event, those neutrons could fake
a neutral current solar neutrino event.
In SNO+ this is not as much a problem because neutron captures in water will
primarily produce a $2.2$\,MeV gamma, which is below the analysis threshold for
solar neutrinos.
However, there does exist events in the dataset which are observed to follow
after high-nhit, events. The origin of these events is not well understood, they
could likely be instrumental, or from  spallation products within the detector.
Since solar neutrino events are not expected to have any time correlations
with other events in the detector, a cut can be placed on the time between
events with relatively little sacrifice.\\
TODO....addmore


\subsubsection{CAEN Cut}
I developed a new data cleaning cut, called the ``CAEN Cut'', that follows from
the AMB Cut from SNO.\@
The AMB Cut attempted to remove events from flashers the dataset by requiring
that the integral and peak height of the ESUMH trigger sum (as measured by the
AMB) fall below some threshold value.
The CAEN Cut performs a similar function, it calculates the baseline subtracted
integral and peak height of the digitized ESUMH trigger signal and places a cut on those values.

The baseline value of each trace is calculated as the average value of the first
$20$ samples and the $65^{\text{th}}$ to $85^{\text{th}}$ samples.
I chose to use two windows, one before the trigger pulse, one after the trigger pulse, to
correct for any overall slope across the digitized window.
The CAEN window is $104$ samples long, the final 19 samples are not used because they often
include a large noise pulse.
The noise pulse comes from the GT pulse arriving at the front-end and generating electrical noise,
it's typically called ``readout noise''.
The readout noise makes the last $\approx20$ samples of the CAEN trace nearly
useless.

The determined baseline is subtracted from the CAEN trace and the integral and maximum
peak height are calculated from the samples between the two baseline windows.
To pass the CAEN Cut the peak and integral must fall between an upper and lower, nhit dependent,
cut cut value.
The cut values are given by
\begin{equation}
    f(n) = C\left(1-\sigma(n)\right) + \sigma(n)\left(mn+b\right)\text{.}
    \label{eqn:cc_threshold}
\end{equation}
Here $\sigma(x)$ indicates a sigmoid function,
\begin{equation}
    \sigma(x) = \frac{1}{1+e^{\frac{-(x-x_{0})}{w}}}
\end{equation}
The cut values are meant to be constant value at lower nhit, and then
linear with nhit above $\approx15$ nhit, the sigmoid allows for a smooth
transition between those two functions; for both the upper and lower threshold
the sigmoid position ($x_{0}$) and width ($w$) are $15$\,nhit and $5$\,nhit respectively.
The constant value at lower nhit is $C$ the slope of the line at higer nhit
is given by $m$ and the value $b$ is required to be
\begin{equation}
    b = \frac{C}{mx_{0}}
\end{equation}
so that there is not discontinuity between the two cut regions.
The values for these parameters are given in Table~\ref{XXX}.

The reason for the two cut regions is because at lower nhit the signal
peak is smaller than the noise one the ESUMH signal, so the only requirement
is that the peak and integral be consistent with a noise only trigger sum.
At higher nhit the ESUMH signal scales linearly with nhit, each new hit
adds approximately the same amount of height to the trigger pulse.

The cut parameters were determined from two calibration datasets, the first was
tagged $\ce{^{16}N}$
events. The second was a sample of $PULSE\_GT$ triggers taken during normal
running.
The two datasets are used to determine the cut parameters for the two
different cut regions.
The $\ce{^{16}N}$ data was used to determine cut values for the higher nhit
region, the $PULSE\_GT$ data was used for the lower nhit cut values.

For both regions the value of the integral or peak height that include
99\% of the events at each nhit is found. Then the parameters of
Eqn.~\ref{eqn:cc_threshold} that best fit those points is determined.
Then Eqn.~\ref{eqn:cc_threshold} with the best fit upper and lower
parameters to include 99\% of the calibration data become the threshold
values for rejecting flasher events.
The 99\% criteria was chosen arbitrarily to ensure that the fraction of
``good'' events rejected by this cut was similar to that of other data
cleaning cuts.
Figure~\ref{XXX} shows how the ESUMH CAEN trace integral is distributed
in the two calibration datasets and for standard physics data taking.



%TODO move this shit to an appendix
\subsection{ZeroZero Cut}
The GTID for the FEC is stored in a ripple counter, it's often the case that
when the bottom two bits of the counter rollover the event that gets recorded in
the FEC memory gets corrupted.
When this happens the builder cannot put the corrupted hits into the event correctly,
and the hits will effectively be discarded.
This means that event the detectors effective photon detection efficiency is lower
for events that have a GTID with $00$ in the bottom two bits.
Rather than correct for this inefficiecy in reconstruction, events with GTID
ending in $00$ are discarded. This corresponds to a random pre-scale
on our by a factor of $\frac{1}{256}$.
\subsection{Crate Isotropy Cut}
The Crate Isotropy Cut is designed to remove events that are isolated
in one or a few electronics crates.
Events originating from light within the detector are unlikely to have any
preference in electronics space.
However hits caused by electrical noise that was created near the electronics
can show a very distinct preference for one crate.
The criteria for this cut is that fraction of hits in any single
is greater that $70\%$ and that the fractions of hits within
that crate are either $80\%$ within adjacent FECs or $70\%$ within
adjacent channels.
%\subsection{Flasher Geometry Cut}
%\subsection{ITC Timespread Cut}

\section{Analysis Cuts}
The dataset of events passing all data cleaning cuts is further reduced by
requiring all events pass cuts on reconstructed quantities.
The cuts are designed to minimize the number of events in the dataset from
non-solar interactions.

Necessarily, the first of these cuts is the requirement that the reconstruction
fits produce to a valid position, time and energy.
The reconstruction algorithm's can fail to converge if an event occurs in an
optically complicated region of the detector, \textit{e.g.} near the detector
neck. The fitting algorithms rely on the assumption that the majority of
the produced light will travel directly from the event vertex to PMT array.
For events in optically complicated regions this assumption is not a good one.
These regions modelled in simulation, so the monte-carlo simulation estimate of
the efficiency of reconstruction to produce valid fits for solar neutrino events
is used.

A cut, called the ``trigger efficiency'' cut,  is placed on the number of
in-time nhits in each event.
This cut ensures the dataset occupies a region where the detector trigger
efficiency is well understood and near 100\%.
This cut ensures the analysis is minimally effected by uncertainties
associated with the detector's trigger.
As mentioned in Sec.~\ref{XXX}, the detector trigger threshold were adjusted
part way through data-taking. The in-time nhit cut was adjusted to
account for this.
For the first trigger period all events were required to have an in-time
nhit greater than or equal to $23$; for the second trigger period this
threshold was reduced to $10$.
This cut is similar to an energy cut because nhit is the best energy estimator
for an event. But, as discussed below, events passing the analysis energy cut
are very unlikely to fail the trigger efficiency cut.

The next analysis cut is a fiducial volume (FV) cut that requires all events
be within $5.3$\,m of the center of the detector.
This cut is designed to reduce the background from radioactive decays within
the AV or from the water outside the AV\@.
For a time period in the data taking an elevated level of radioactive
backgrounds was present in the upper half of the external water.
During this period the FV cut was modified to require any events in the
top half of the detector ($z > 0$) fall within a radius of $4.2$\,m,
events in the bottom half of the detector were still subject to the standard
$5.3$\,m FV cut.
Figure {XXX} shows the expected distribution of background events from
AV and external water backgrounds compared to the distribution for solar neutrino
events. This more restrictive cut was in place for 13\% of the dataset livetime,
runs XXX to XXX\@.

Two event-quality cuts are further placed on events. These cuts remove events
that have an ITR or $\beta_{14}$ (described in Sec.~\ref{sec:ITR} and Sec.~{sec:b14})
that is inconsistent with the event originating from Cherenkov light produced
by a solar neutrino event.
The ITR of events is required to be greater than $55$\% and the $\beta_{14}$
value must between $-0.12$ and $0.95$.
These cuts are similar in purpose to the data cleaning cuts, as they attempt to
remove events that are produced by sources other than Cherenkov light within
the detector.

The final cut analysis cut is on the reconstructed kinetic energy of each event.
The energy region for this analysis is $5.0 > T_{\mathrm{e}} < 15.0$\, MeV.
This region is chosen to minimize contamination from atmospheric neutrino
interactions, and radioactive decays within the detector.
Additionally, the only solar neutrino flux that is significant across this energy range
are the neutrinos from the $\ce{^{8}B}$ solar reaction.
Neutrinos from the $hep$ interaction also fall within the same energy range,
however their flux is expected to be much lower than that of the $\ce{^{8}B}$
neutrino flux that their presence can be largely neglected.

In principle the lower energy threshold could be lowered or removed to
increase the fraction of solar neutrino events in the dataset, however,
the rate of backgrounds from radioactive decays increases rapidly at lower
energies. So no additional sensitivity to the solar neutrino interaction
rate would be gained with a lower energy threshold.
The $5$\,MeV threshold was chosen as the lowest energy from which a solar
signal could still be resolved.

A summary of all cuts and their criteria, where applicable, is given in Table~\ref{tbl:event_selection}.

\begin{table}
    \centering
  \begin{tabular}{c  c}
        Energy & FV\\
    \end{tabular}
    \caption{Cuts}
\label{tbl:event_selection}
\end{table}

\section{Livetime}
While most event selection removes individual events based upon whether they
pass or fail certain criteria, some cuts remove all events that occur for a
period of time before or after some criteria is met.
These cuts are said to introduce a deadtime into the dataset.
The most significant example of this is the muon follower data cleaning cut, which
cuts all events for 30\,seconds after every muon interaction in the detector.
The livetime for each run is then defined as
\begin{equation}
    t_{\mathrm{live}} = t_{run} - t_{\mathrm{dead}}\text{,}
\end{equation}
where $t_{run}$ is the time between the first and last valid event within a run
and $t_{\mathrm{dead}}$ is the sum of all deadtimes introduced into that run by cuts.
The livetime is used to calculate the total exposure represented by the dataset.
Table XXX shows the sum of deadtimes across all runs within the dataset.
For simulated events many of the effects that necessitate deadtime are not simulated,
so no deadtimes are added into the simulated runs.
