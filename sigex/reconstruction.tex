\section{Reconstruction}
A series of reconstruction algorithms are ran over all events that pass data cleaning.
These algorithms estimate the position, time, direction, and energy of the event.
All events are reconstructed under the hypothesis that the PMT hits are from cherekov radiation
produced by a single electron.
Additionally, the reconstruction algorithms use only the hits in the prompt time window to ensure only light
that travelled directly from the event origin is used.
The same reconstruction algorithms are used on both simulated and detected events.

The direction ($\vec{d}$), time ($t_{0}$), and poition ($\vec{p}$) are determined by performing a likelihood
fit to the time and position of PMT hits.
The algorithm evaluates the likelihood of a hypothesized event position and time by
calculating the time residual for each hit PMT,
\begin{equation}
    \label{eqn:tres}
t_{res} = t_{PMT} - t_{transit} - t_{0}
\end{equation}
and using a PDF for $t_{res}$ determined from simulation, $P(t_{res})$.
The position and time that minimize the quantity
\begin{equation}
\sum_{i=0}^{N_{PMT}} P(t_{res}) % TODO need to refactor this to include the index i
\end{equation}
is used as the event position and time.
The direction is determined by evaluating $\theta_{PMT}$ for each hit where
$\theta_{PMT}$ is defined by,
\begin{equation}
    \cos\theta_{PMT} = \vec{d}\cdot\left(\vec{p}_{PMT} - \vec{p}\right)\text{.}
\end{equation}
The likelihood, $P(\theta_{PMT})$, is determined from simulation, the direction
that minimizes
\begin{equation}
\sum_{i=0}^{N_{PMT}} P(\theta_{PMT})
\end{equation}
is used as the reconstructed event direction.

The kinetic energy of the event is determined separately
using the best fit position, and time as an input.
The position and time are used to determine the number of PMT
hits that occurred in a prompt 18\,ns window.
Then the number of photons that would most likely produce
that number of PMT hits is estimated using a combination of
analytic calculation and monte-carlo simulation.
A look up table is used to estimate the most likely electron
kinetic energy that would produce the determined number of photons.
This method of energy reconstruction is called ``EnergyRSP'', which stands
simply for Energy Response.

Figure XXX shows the residuals for fit results on MC simulated events.

%An error was found in the calculation of the number of Cherenkov photons expected for an electron of a given energy.

\subsection{ITR}
The time residual, defined in equation{eqn:tres} for a PMT hit is an extremely useful quantity because in
general light that travels directly from an interaction will have a very small
time residual.
Light that is produced by another source, or reflects off of a detector component
between production and detection will have a larger time residual.

The fraction of hits that satisify
\begin{equation}
    -4 > t_{res} < 9 %TODO CHECK VALUES
\end{equation}
is known as the ``In-time ratio'' (ITR).
The expected distrubtion in ITR for electrons is shown in figure XXX.

\subsection{$\beta_{14}$}
The quantity $\beta_{14}$ is used to quantify how isotropic the hits
in an event is. It is defined as
% TODO check this is correct
\begin{equation}
    \beta_{14} = \sum_{j=0}^{i}\sum_{i=0}^{i=N_{PMT}} P_{1}(\cos(\theta_{ij})) + P_{4}(\cos(\theta_{ij}))
\end{equation}

% TODO, check this as well
The quantity $\theta_{ij}$ is the angle subtended by the vectors pointing
from the reconstructed position of the event to the $i^{\text{th}}$ and
$j^{\text{th}}$ hit PMT.

The expected distribution of $\beta_{14}$ for electron events within the
detector volume is shown in Figure XXX.

%The fraction of hits that appear in a very prompt time window is a useful
%value for removing instrumental backgrounds. Cherenkov radiation produces
%a very short pulse of light within the detector, so any source of light
%that has a longer time profile is unlikely to be from a Cherenkov process.
%This is quantified using the ``In-time ratio'' (ITR) for an event.
%ITR is the fraction of light with a time residual.
