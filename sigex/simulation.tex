\section{Simulation}


\subsection{RAT}
A monte-carlo simulation of particle interactions in the detector is used
for predicting detector observables for events.
The simulation package used is called RAT, it is a Geant4-based simulation that
contains a detector and DAQ simulation in addition to simulation of particle
interactions and photon propagation.


\subsection{Solar Neutrino Fluxes}
The expected spectral shape and normalization for the solar neutrino signal is
taken from the BS05(OP) standard solar model~\cite{XXX}.

\subsection{Solar Neutrino Cross-sections}
The rate of solar neutrino for a given flux follows from the cross-section for
interaction. The only interaction relevant for the SNO+ detector is the
neutrino-electron elastic scattering interaction.
%Neutrino-nuclear interactions occur as well, but they either cannot be identified
%separately from the radioactive backgrounds, or they occur at a very small rate
%compared to the elastic-scattering rate.

The cross-section for the elastic-scattering interaction is take from Bahcall
\textit{et.\al}~\cite{escrosssec} which includes radiative corrections to
the cross-section.

\subsection{Survival Probability Simulation}
A simulation of the expected solar survival probability curve for a given
set of mixing parameters was used.
The survival probability is calculated using a
3-flavor adiabatic calculation. The calculation was developed by the
SNO collaboration~/cite{XXX}.
is used to calculate the fraction of the solar neutrino flux that arrives
at Earth and interacts as a $\nu_{e}$ vs the fraction that is $\nu_{\mu}$ or $\nu_{\tau}$.

