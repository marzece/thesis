\section{Simulation}
\subsection{RAT}
A monte-carlo simulation of particle interactions in the detector is used
for predicting detector observables for events.
The simulation pacakage used is called RAT, it is a Geant4-based simulation that
contains a detector and DAQ simulation in addition to simulation of particle
interactions and photon propagation.

\subsection{Solar Neutrino Fluxes}
The expected spectral shape and normalization for the solar neutrino signal is
taken from the BS05(OP) standard solar model~\cite{XXX}.

\subsection{Solar Neutrino Cross-sections}
The rate of solar neutrino for a given flux follows from the cross-section for
interaction. The only interaction relevant for the SNO+ detector is the
neutrino-electron elastic scattering interaction.
%Neutrino-nuclear interactions occur as well, but they either cannot be identified
%separately from the radioactive backgrounds, or they occur at a very small rate
%compared to the elastic-scattering rate.

The cross-section for the elastic-scattering interaction is take from
XXX.
XXX Something about radiative corrections

\subsection{Survival Probability Simulation}
A simulation of the expected solar survival probability curve for any set of mixing
parameters was used. The survival probability is calculated using a
3-flavor adiabatic calculation. The calculation was developed by the
SNO collaboration \cite{XXX}.
is used to calculate the fraction of the solar neutrino flux that arrives
at Earth and interacts as a $\nu_{e}$ vs the fraction that is $\nu_{\mu}$ or $\nu_{\tau}$.

\section{Analysis}
The solar neutrino flux present in the dataset is determined by first
rejecting events from the dataset that are unlikely to be solar neutrino events.
Events that pass all cuts are two-dimensionally histogrammed in kinetic energy $T_{e}$ and
$\cos\theta_{sun}$.
The range of energies considered is $5\text{\,MeV} > T_{e} < 15\text{\,Mev}$.
That range of energies was chosen to minimize contamination from radioactive
backgrounds and atmospheric neutrino interactions.
Between energies \numrange{5}{10}\,MeV the histogram bin width
is 1\,MeV, above 10\,MeV a single bin is used.
Forty bins of equal width are used for histogramming events in $\cos\theta_{sun}$.

Simulated events are used to estimate the expected distribution of events in
$\cos\theta_{sun}$ and electron recoil energy $T_{e}$.
Simulated events are treated the same was as detector events, the same
cuts are applied to their reconstructed quantities.

A position dependent correction is applied to the reconstructed energy
of both MC simulated events and detector events.


The simulated events are histogrammed to estimate the underlying PDFs
of observable reconstructed energy and event direction. Cut's are placed
on each event to

