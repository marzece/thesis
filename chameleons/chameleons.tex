
%: ----------------------- sigex file header -----------------------
\chapter{Chameleons}

% the code below specifies where the figures are stored
\ifpdf
    \graphicspath{{chameleons}{chameleons/figures/PDF/}{chameleons/figures/}}
\else
    \graphicspath{{chameleons/figures/EPS/}{chameleons/figures/}}
\fi

% ----------------------------------------------------------------------
% ----------------------- chameleons content -------------------------
% ----------------------------------------------------------------------

The observed discrepancy in $\Deltam^{2}_{21}$ has motivated a number
of theories that modify solar neutrinos oscillation from the standard
MSW-LMA hypothesis described in section $XXX$.
Here I'll describe a few of those theories and introduce a novel theory
that modifies the potential the neutrino experiences as it travels in
vacuume between the Sun and Earth.

\subsection{Non-Standard Interactions}
Solar neutrino moving through the core of the sun is one of the few sources
of neutrinos that experience oscillations that are significantly modified by
the ambient electron density.
In princible neutrino mixing could be modified by neutrino-nuclear interactions
as well, however standard nuclear interactions for the neutrino are either
not flavor sensitive. Or are the result of incoherent neutrino scattering, which
in most cases has a much smaller  cross-section than coherent or electron scattering.


But their potential sensitivity to nuclear or modified electron scattering
means they can be used to probe into our understanding of how neutrinos
interact with matter.
If there exists neutrino-nucleus or neutrino-electron interactions that are
not accounted for within the standard model those interactions could be visible
in how they modify the oscillations of solar neutrinos.


It's common to parameterize these modified interactions in a general manner,
without tieing it to particular theory of modified interactions. As show in
figure XXX, which shows a possible survival probability curves for a modified
up or down quark-neutrino interaction\ldots TODO.\@

\subsection{MaVaNs}
It was originally proposed in XXX that the cosmological acceleron theory for dark
energy might lead to a neutrino's mass being modified by the local neutrino density.
This in turn can modify neutrino mixing in the core of the sun where the neutrino
propagates significant distances in areas of high neutrino density.


\subsection{Chameleons}
It was proposed in XXX that the observed expansion
of the universe could be explained by introducing a 5th force that is weak
in areas of high matter density. Such a force would be difficult to detect
in most experiments because its effects would be small compared to standard
forces. But at cosmological distance scales where the matter density is near
zero, the force could be much stronger. This force is referred to as a ``hameleon''
force, because it is affected by it's surrounding and can ``blend in'' to avoid
detection.

It could be the case that this force couples to neutrinos such that the coupling
is sensitive to either neutrino flavor, or the mass. If so it's expected that the
neutrino's mixing would be modified by the prescesce of a chameleon field.
And that the chameleon-modified mixing would be only significant in areas of very
low matter density.
Solar neutrinos would provide almost unique sensitivity to this sort of modified
vacuum mixing, because they're by far the most abundant and easily detectable source
of neutrinos that travel a significant distance in areas of near-zero matter
density, \it i.e.\ between the Sun and Earth.

This idea is the phenomonlogical basis for the idea of modified vacuum mixing.
To explore this idea I developed a simulation of neutrino mixing in the sun and in
the vacuum. Typical calculations of the neutrino survival probability from
the sun are able to take advantage of the fact that the neutrino travels adiabatically
through the varying mass density of the sun. This means that the neutrino is
created in a mixture of mass states, the exact composition depending on the local
electron density, and the neutrino stays in that same mixture of mass states
as it exits the sun. The flavor composition of each mass state however changes,
meaning that the flavor composition of the neutrino state changes, even though
it's mass-state composition does not.
This means that practically all one needs to do is calculate the composition
of mass states that corresponds to an electron type neutrino, and then calculate
the flavor content of that composition in vacuume, and that flavor content
tells you the survival probability and trasition probability. One is able to
ignore all oscillations that may occur within the sun and simply calculate
quantities for when the neutrino is created and when it is detected.

This simplification is not necessarily valid depending on how the neutrino
exits the sun and how quickly the modified vacuum potential becomes significant.
While the matter density is low but non-zero the neutrino will experience standard
vaccuum mixing. If the neutrino changes from low to zero matter density over
a distance much shorter than the oscillation length of the neutrino, the
transition from standard to modified vacuum mixing potential may not be adiabatic.
In which case the exact state of the neutrino at the point of the transition
will determine how the flavor state of the neutrino changes as it propagates.

To allow for this the simulation for modified vacuum mixing simulates the full
neutrino state as it propagate through the sun. The equation
\begin{equation}
    \label{eqn:mixing_schrodinger}
    i\frac{d\Psi}{dx} = H\Psi %TODO
\end{equation}
is evaluated numerically using the Runge-Kutta method of numerical integration.
This cannot be easily evaluated analytically due to the varying density in the
core of the sun.

Performing this simulation for many neutrino energies at many different starting
radii gives an representation of possible neutrino states for solar neutrino.
The density profile for the sun is taken from XXX.%todo

The standard solar survival probability can be calculated by taking calcuating
$|\bra{\nu_{e}}\ket{\Psi_{\nu}}|^{2}$ for each simulated neutrino and appropriatly
averaging over energies and production radii.
Since neutrino states are simulated by linearly sampling starting radii and logarithmically
sampling neutrino energies, states must be weighted by the relevant production
PDFs in energy and radius.

The simulation of neutrinos this way is computationally expensive. A few methods
were explored for ensuring this simulation could be performed in a reasonable
amount of time. The method that was used for nearly all of the results here was
to perform the Runge-Kutta integration on a GPU, which each thread corresponding
to a single sample in energy and production radius.

% TODO need typical radii/energy sample counts
However, even using GPU acceleration the simulation is still very time consuming,
a simulation of XXX energy samples and XXX production radius samples requires
roughly 2000\,gpu-hours. Performing this simulation as part of a fit to data would
require potentially hundreds or thousands of iteration. So it is not possible to
perform the full simulation in a fit.

Fortunately, by construction the solar simulation is not effected by modified
vacuum potential; the main inputs to the solar simulation are the standard model
mixing parameters and the solar density profile. So, standard model mixing parameters
taken from KamLAND and other non-solar neutrino experiments can be used for
the solar simulation.

The result of the solar simulation is the neutrino state at 5000 samples
closest to the exit of the sun. Depending on the energy of the neutrino this
corresponds to state the neutrino is in in the final  150 to 500 km of
the Sun, this corresponds to \numrange{0.2}{0.7}\% of the solar radius ($R_{odot}$).
And the sample-to-sample distance is \numrange{30}{100} meters.
Production radii samples are XXX\,m from each other, meaning that
the \numrange{150}{500}\,km samples taken at the end of the simulation
overlap with samples taken one production radius step further.
This provides a useful check of the simulation, the difference between two
samples which have travelled the same distance within the sun should only depend
on the difference in electron density where they were produced.
Figure shows that correlation\ldots.%TODO!(maybe, not that important)

Once  monte-carlo samples of neutrino states produced by the Sun is calculated, these
states are used as inputs to a simulation of the modified vacuum potential.
This simulation is in princible the same as the solar simulation,
it simply involve evaluting Eqn.~\ref{eqn:mixing_schrodinger}, where $H$ now
corresponds to the modified vacuum Hamiltonian.
Unlike the solar simulation though the value of $H$ is not expected to change
as the neutrino propages; Equation~\ref{eqn:mixing_schrodinger} can be evaluated analytically
between the Sun and Earth.

The final step of the calculation is to evolve the sampled neutrino states through
the Earth, to the detector.
This is done similarly to the simulation of neutrino propagation through the
Sun.
The calculation for this is done for only a ``day'' path through the earth
and a ``night'' path. The ``day'' path simulates the neutrino only travelling
through the crust of the Earth. The ``night'' path simulates the neutrino
travelling through the Earth, including the high density ``core'' region.
The Earth density profile is take from PREM XXX% \refXXX.\@%todo.
This results in an simulation of the day-night effect for neutrino
oscillation.

The final result of this chain of simulation steps is monte-carlo samples
of neutrino flavor states. The survival probability can be calculated by calculating
$P_{ee}(\Psi_{nu}) = |\bra{\nu_{e}}\ket{\Psi_{nu}}|^2$ for each neutrino state.
Performing an average of states and binning in neutrino energy gives the survival
probability as a function of energy $P_{ee}(E_{\nu})$.

Since neutrino states are monte-carlo sampled to calculate $P_{ee}(E_{\nu})$
each value has statistical uncertainty from the number of samples used.
This problem was somewhat exacerbated by the distributions of some of the
solar neutrino energy PDFs having small values in areas that are important for
comparing to solar neutrino data. For example the low energy portion of the
$\ce{^{8}B}$ solar neutrino flux is very important for solar neutrino experiments,
but makes up a relatively small portion of the full $\ce{^{8}B}$ neutrino flux.
To mitigate the problem of large sampling uncertainty for important regions in
solar neutrino energy, energies were sampled according to a flat distribution
and according to the PDFs for each solar neutrino flux. These two methods of sampling
were performed in equal proportions for each flux type.

\subsection{Simplified Modified Vacuum Mixing}
Probing the idea of modified vacuum mixing with the simulation detailed in
section XXX proved computationally difficult.
So I explored simplified method for evaluating the likelihood of a modified
vacuum mixing potential. The simplification was to restrict the modified mixing
to be equivalent to a change in the effective value for $\Deltam^{2}_{21}$.
The motivation being that the observed discrepancy between solar neutrino
experiments and KamLAND was only in $\Deltam^{2}_{21}$, and not in $\theta_{12}$.

To explore this idea one simply can use standard methods for calculating the
survival probability, but modify them such that all terms are effected by the
local electron density ($n_{e}$) use a different value for $\Deltam^{2}_{21}$ than the
terms that are not effected by $n_{e}$.
This introduces a new parameter into the theory, $\Deltam^{2}_{21}'$, the
effective mass-squared splitting the neutrino experiences in vacuum.

With this modification a fit to solar neutrino data was performed, allowing
all mixing parameters to vary. If this version of modified vacuum mixing describes
reality then the best fit value for the matter mass-splitting ($\Deltam^{2}_{21}$) should be
consistent with the value determined by KamLAND.\@ The value for the vacuum mass
splitting ($\Deltam^{2}_{21}'$) has no-apriori preferred value but it would
be sensible for it to be near the standard best fit value for $\Deltam^{2}_{21}$
as determined by solar neutrino only measurements.

The fit to data was performed using a Markov-chain Monte-Carlo method to sample
the likelihood space of mixing parameters as well as solar neutrino fluxes.
Figure XXX shows the results of the MCMC sampling. Marginalizing over all
the mixing parameters, including $\Deltam^{2}_{21}'$, gives the best fit
value for $\Deltam^{2}_{21}$ in matter and the error on it.
The marginalized result is shown in Figure XXX, the preferred value
for $\Deltam^{2}_{21}$ from solar experiments is $XXX\pm XXX$, only slightly higher than
the preferred value in a standard mixing formulation, $XXX$, but still significantly
lower than the best fit KamLAND value, $XXX$.
The tension between the solar and KamLAND values of $\Deltam^{2}_{21}$ is at
the $XXX\sigma$ level in the standard formulation, this version of modified
vacuum mixing reduces that to $XXX\sigma$, at the cost of introducing a new
parameter into the theory.

The improvement in agreement between solar neutrino experiments and KamLAND on
the value of $\Deltam^{2}_{21}$ is not large enough to constitue compelling evidence
that this simple version of modified vacuum mixing describes reality much better
than standard mixing. And so this motivates going back to a fuller description
of modified vacuum mixing, that allows for a fuller description of how
neutrinos might oscillate between the Sun and Earth.
