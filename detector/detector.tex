
%: ----------------------- detector file header -----------------------
\chapter{Detector}

% the code below specifies where the figures are stored
\ifpdf
    \graphicspath{{detector/figures/PNG/}{detector/figures/PDF/}{detector/figures/}}
\else
    \graphicspath{{detector/figures/EPS/}{detector/figures/}}
\fi


% ----------------------------------------------------------------------
% ----------------------- Detector Chapter -----------------------------
% ----------------------------------------------------------------------
% 9385 PMTs comes from doing the following SQL query on DB
%select count(*) from pmt_info where (type & 1)>0 and (type & 32)=0  and (type & 128) =0 and (type & 2)>0 and (type& 64)=0 and (type & 16)=0 and (type & 8)=0;
\section{The SNO+ Detector}
\subsection{Detection Mechanism}

The primary neutrino interaction that SNO+ is sensitive to is elastic scattering
off electrons,
$\nu_{x} + e^{-} \rightarrow \nu_{x} + e^{-}$
where $x=e$,$\mu$, $\tau$.
For $x=e$ there exists a neutral current and a charged current channel,
for $x=\mu$ or $x=\tau$ there exists only the neutral current channel.
% TODO try and remember what actually I mean here
Nuclear interactions occur as well with the oxygen in the water,
however these are rare and difficult to identify, and so are ignored typically.
The elastic scattering cross-section is given by
\begin{equation}
cross section
\end{equation}
This can also be expressed as it's differential cross-section as a function
of energy
\begin{equation}
    d\sigma/dT_{e}
\end{equation}
and of scattering angle
\begin{equation}
d\sigma/d\theta
\end{equation}
These cross sections are shown in figure blah.
The relevant points are that the angular cross section is peaked in
the forward direction, meaning that information about the neutrino direction
is maintained in the interaction.
But the differential cross-section for recoil energy is nearly flat below the
end point. Meaning relatively little information about the incoming neutrino energy
is preserved by the interaction.
For the SNO+ water-phase the scattered electron generates light via Cherenkov radiation,
assuming it's above Cherenkov threshold for water which is $XXX$\,MeV.
If it is above the Cherenkov threshold the electron will generate photons that
travel at approximtely a 42$\deg$ angle to the direction of the electron.
The angle of travel comes directly from the speed at which electro-magnetic
signals propagate in the medium.
Figure $TODO$ shows this diagramatically, as the charged particle (in this case
an electron) distance $\ell$ from point A, to point B, the electro-magnetic wave
that was emitted at A will travel a distance $\ell \dfrac{c}{v}$, forming
a spherical wavefront at that distance.
When the electron travels another distance $\ell$ from B to C the wavefront
from A is a distance $2\ell\dfrac{c}{v}$ from A and the wavefront from B is
a distance $\ell \dfrac{c}{v}$ from B.


%%% TODO it'd be dank to have a chapter discussing Cherenkov radiation and deriving
%%% the wavelenght and angular distribution
%\includegraphs{cherenkov_radiation_cone_pictures}
The photons can the be detected by the SNO+ PMT array, and the pattern
of hits analyzed to determine the electron direction, energy, and position.

SNO+ has $XXX$ inward looking PMTs all mounted on a geodesic spere referred to as
the PMT support structure (PSUP). The PMTs are at an average radius of 8.4\,m from
the center of the detector.
Mounted on the outside of the PSUP are 90 outward looking (OWL) PMTs.
These serve to reject interactions in the outer volume from cosmic muons.
All PMTs are Hammatsu R1408 8-inch PMTs, which have typical quantum efficiency
of  $XXX$\%.

The inward looking PMTs are all housed within a plastic casette.
Each PMT is also $XXX$collared$XXX$ by an array of reflective
petals which serve to increase the effective photo-sensitive area
of the PMT.
The geometric coverage of the PMTs is $XXX$. % TODO, see if you can get the effective coverage



\subsection{The Detector in Brief}
The SNO+ detector can be mostly simply described as a large volume of some
target material that is deep underground and is observed with an array of
9385  photomultiplier tubes (PMTs).
Changing the target material can change what sort of physical processes is
observable with the detector.
The SNO detector was originally designed for a heavy-water ($\ce{^{2}H_{2}O}$ or $\ce{D_{2}O}$)
target, the upgrades for SNO+ were done in expectation of a tellurium doped
scintillator target, more information on this can be found in Sec.~\ref{sec:upgrades}.
For this thesis the target material is ultra-pure water (\ce{H_{2}O}).

The target volume is encapsulated within a 6-meter radius acrylic sphere,
which is held suspended in a large cavity filled with UPW.
The acrylic sphere has a XXX meter acrylic chimney, called the ``neck'', at its top
to allow access to the detector volume.
Surrounding the acrylic vessel (AV) is an  array of inward pointing PMTs
arranged in a geodesic pattern.
The structure holding these PMTs is referred to as the PMT Support Structure
(PSUP).
There are roughly 90 PMTs on the PSUP that point outward, toward the cavity
walls.
These outward facing tubes are called OWLs and are for the purpose of tagging
interactions that occurred outside the PSUP.
There are an additional three tubes mounted at the top of the neck of the AV,
these are referred to as NECK tubes.

Above the cavity volume is an optically isolated deck on which all the detector
readout and trigger electronics are kept.
PMTs are housed within a water-tight cassette and readout via a custom BNC-like
connector that connects to the PMT interface electronics.
Within the PMT housing is a custom PMT-base that fans out the approximately 2kV
high voltage (HV) supply to the PMT input pins and routes the PMT return signal to the same
HV supply cable.


\subsection{Electronics And DAQ}
The SNO+ data acquisition (DAQ) inherit much of its design and components from
SNO.
There are a few notable upgrades that were made for the purpose of handling the
higher light yield and event rate that SNO+ has compared to SNO.
The DAQ hardware can be described as a few separate systems, the trigger system,
the readout system, and the PMT interface system.
The PMT interface provides an approximately 2\,kV (HV) supply to each PMT and provides
the signal from the PMT to the rest of the DAQ electronics.
The trigger system's purpose to decide when an interesting interaction
within the detector has occurred, and to start the readout process when such an
interaction has occurred.
The readout process is responsible for ensuring enough information about each
PMT signal is recorded such that offline analysis is possible.

The first step of the PMT interface system is the PMT base. The base is responsible
for fanning-out the supplied HV to the PMT dynode pins and connecting the PMT output
to a PMT cable. The PMT base is housed within a water-tight cassette.
The PMT cables pass through penetrations in the cavity ceiling where they then connect
to the rest of the electronics. The PMT cable connects first to a PMT interface
board (PMTIC). The physical connection occurs on a daughter card, called a ``paddle card'',
that accomadates up to eight PMT cables; Each PMTIC hosts four paddle cards.
The PMTIC is responsible for fanning out the PMT high voltage
to each PMT and providing channel level adjustment to the voltage
each PMT receives; the voltage adjustment is done with a series of
swappable resistors.
The PMTIC is also responsible for separating the PMT signal from the supplied
HV, this is achieved with a capacitative decoupling circuit.
Once the two signals are separated the PMTIC sends the PMT signal to
a front end card (FEC) via a board-to-board connector, where it enters
the readout and trigger system.

%The FEC sends each PMT signal to one of four mounted daughter boards
%(DB). Each DB compares the PMT signal to a voltage threshold to decide if a "hit" has
%occurred or not; thresholds are set on a channel-by-channel basis.
%When a hit occurs a number of processes take place to record information and
%provide inputs to the trigger system.
%The inputs to the trigger system are four separate analog current pulses:
%N100, N20, energy sum high (ESUMH) , and energy sum low (ESUML).
%The N100 and N20 are fixed height square pulses that have different widths.
%The N100 is near 100\,ns wide, the N20 is nominally 20\,ns wide but for nearly
%all of the data taken with detector it was changed to be 40\,ns wide.
%The ESUMH and ESUML pulses are copies of the original PMT pulse, with different
%gains applied - ESUMH having a larger gain than ESUML.

That signal is compared to a threshold, if the signal is over threshold a "hit"
has occurred\,-\,this threshold is often called the ``channel threshold''.
At the time of the channel threshold crossing the following processes occur, the name for each is given
in parenthesis:
a 100\,ns long fixed-height square pulse is created (N100), a 20\,ns fixed height square pulse is created (N20),
a high gain copy of the signal is created (ESUMH), a low gain copy of the signal
is created (ESUML), a linear voltage ramp begins (TAC ramp), the signal is integrated for
50\,ns with high gain (QHS), the signal is integrated for up to 400\,ns with high gain (QHL),
and the singal is integrated for 50\,ns with a low gain (QLX).
These signals and values are created on a few different custom ASICs on the
daughter boards.
The trigger system uses the first of those 4 signals (N100, N20, ESUMH, and ESUML).
The readout system uses the latter four values (TAC, QHS, QHL, QLX).

The trigger signals are all combined with their counterparts from
other PMT channels across the detector, \textit{i.e.} the N100 signals
from all channels with be combined and separately all the
N20 signals will be combined, \textit{etc}.
The signals are combined through analog summation, summing is done on a few different
circuit boards within the detector.
The FEC sums the top and bottom sixteen channels separately, the crate
trigger card (CTC) sums the signals from the sixteen
FECs that are in each electronics crate.
The signals from each of the nineteen CTCs are all summed on the
Master Trigger Card - Analog (MTCA+). The SNO+ MTCA+ is an upgraded
version of the SNO MTCAr; more information about the MTCA+ is available in
Sec.\,\ref{sec:mtcap}.

Separate, but identical, MTCA+s are used for each of the four trigger signals.
Each MTCA+ performs the analog summation with three different gains,
resulting in a total of twelve signals spread across four different boards.
Each of the twelve signals are separately compared to a threshold;
each of the twelve thresholds are independent from each other.
These thresholds are called "trigger thresholds".

The different gains are in place due to the practical difficulty of maintaining
a good signal-to-noise ratio (SNR) without limiting the range of the
system.
For example, if there exists 10\,mV of noise in the system a 20\,mV pulse
would give a 2:1 SNR, however this would mean if 5000 PMT hits occurred simultaneously
the signal would be 100\,V in size.
It is not practical to have a system with 100+\,V range and 20\,mV resolution,
so the three different gain paths allow for three different trade-offs between
SNR/resolution and range.
The highest gain signal has the best SNR, but the smallest range, and so usually
the highest gain signal has the lowest effective threshold.
The reason being that it's more important to have single hit resolution at a threshold
of 8-hits than it is at a threshold 25 hits.
The different gains on each signal are therefore labelled by their threshold (not their gain), e.g.
the high, medium and low gain paths for the N100 signal are respectively called
N100 Low (N100L), N100 Medium (N100M), and N100 High (N100H).

Although there are twelve signal-gain combinations available only seven are
used: N100-Low, N100-Med., N100-High, N20-Low, N20-Med (also called just N20), ESUMH-Low, and ESUML-Low.
Since the ESUMH and ESUML each only use one gain path, they're usually
referred to simply as ESUMH and ESUML with their gain path understood to be
the high gain path.

When a trigger signal goes over its threshold a 20\,ns digital pulse is
emitted for that signal. This pulse is called a "raw-trigger" and there is
one for each of the seven used trigger signals.
The raw-trigger signals are sent from the MTCA+s to the Master Trigger
Card - Digital (MTCD).
Finally, each of these seven raw-trigger signals can be masked in or masked out on
the MTCD;
if a raw-trigger is masked out, nothing happens when it fires,
if it is masked in, then the raw trigger creates a "global-trigger" (GT) signal.
That global trigger signal is fanned out to all the data crates which
in turn sends the GT to all front end cards and daughter boards.
As the GT signal is created the MTCD also generates a signal
called Lockout (LO). Lockout is typically a 420\,ns long pulse and while
the signal is high the MTCD will not create any more global triggers.

Once the global trigger is created the trigger cycle is complete and
the readout process begins.
The raw-trigger signal that caused the global trigger, as well as any other
raw-trigger signals that were high within a 20\,ns window of the global trigger,
are recorded and readout, this is know as the "trigger word".
When the GT is created a counter, called the global trigger identifier (GTID) is incremented
and readout along with the trigger word.

The four values that are created by the PMT signal crossing
the channel threshold (TAC, QHS, QHL, QLX) are stored in analog memory
cells on the daughter boards.
They are stored for a length of time known as ``GT\_VALID'', if
a GT does not arrive before GT\_VALID expires the TAC, QHS, QHL, \& QLX values
are discarded. A typical value for GT\_VALID is $400\,ns$, although
there exists some channel\,-\,to\,-\,channel variation.
If a GT signal does arrive at the channel before GT\_VALID expires the
values in the memory cells are digitized and readout to a memory buffer
on the FEC.
The TAC ramp starts when the PMT signal crosses channel threshold
and stops when the GT signal arrives at the channel.
Since the TAC voltage ramp is linear over time the value of the TAC
indicates when the hit occurred relative to the GT signal.

The FEC stores those values and adds information to identify which
channel's data is stored, it also records the value of its own GTID.
Each FEC keeps a counter that is incremented every time it recieves a global
trigger signal, in princible the value of this counter will always be the same
as the MTCD GTID, and the same as the counter in every other FEC in the detector.
The GTID counter is our only way of associating recorded hit data with each other
and with the trigger word.

In practice it's possible for a channel's GTID to become out of sync with the GTIDs of
all other channels.
This can result in the hits on a particular channel being associated with
the wrong event.
To mitigate this problem every $2^{16}$th and $2^{24}$th GT respectively creates a $SYNC$ and
$SYNC24$ signal, those signals are sent by the MTCD to each FEC \& DB.
If a FEC or DB receives either of these synchronization pulses but its own GTID counter is not
at an increment of $2^{16}$ or $2^{24}$ then the channel is identified as out of sync.
If this happens, the GTID counter is adjusted to the correct value and the next hit to read out from the out of sync channel(s) is accompanied
by a flag to indicate that it was out of sync.
This system ensures a channel is never out of sync for more than $65536$ events.

A short while after the data and the associated identifying information and status flags are buffered
in FEC memory, the data is readout by a create level readout card,
the ``XL3''.
The XL3 is new to SNO+; it replaces the XL1 and XL2 from SNO, more will be said about the
XL3 in Sec.\,\ref{sec:xl3}.
The XL3 reads out each FEC in sequence across the VME-like ``SNOBUS'' backplane.
The XL3 stores data in it's own memory until eventually reading it out over
ethernet to data-server process running on a near by computer.

Each data crate has its own XL3, all XL3 read out and serve data asynchronously.
The data-server process receives data from each XL3 and relays that data to
any clients that have subscribed to the PMT data feed.
A similar process is done for the trigger word data. The MTCD sends trigger data
to the data-server, the data-server relays that data to any clients that have subscribed
clients.

The primary client to the data server is what's known as the ``Event Builder'', sometimes
called the EB or just the ``Builder''. The Builder receieves data from the data-server and
uses GTID information to associate trigger words and hits with each other.
Once all the hits for an event have been associated wiht their trigger word the event
has been ``built'' it is written to disk and the read out process for that event is complete.
Data is typically taken in hour long chunks referred to as a ``run'';
every run has an unique number associated with it and a ``run type'' number that
gives basic context to the detector circumstances and settings in which the data was taken.
The Builder, in addition to building events, is reponsible for associating
events with their run number and run type.

There are a few ancillary systems within the DAQ electronics, all
of which are new to SNO+.
The first is the CAEN v1720, commonly
referred to as just ``CAEN'', which is a 12-bit digitizer board.
The CAEN is used to digitize and readout the trigger signals.
It has eight available input channels that it can digitize, however,
typically only three signals are actually used, those channels digitize ESUMH, N20L, and
N100L. The CAEN's digitization window and sampling rate can be varied,
most commonly the digitization window is 420\,ns and the sample width
is 4\,ns. The CAEN receives a copy of the global trigger allowing and
it keeps it's own GTID counter so its data can later be associated
with the appropriate hit and trigger data. It also receives a copy of the
SYNC and SYNC24 signal so it's synchronization can be ensured.

The input voltage range for the CAEN is an adjustable 2\,V window.
The voltage range for the trigger signals is 10\,V.
The difference in ranges necessitates some way of reducing the range of the trigger signals
before they're sent to the CAEN.
The simplest way of reducing the voltage range is to use a voltage
divider to attenuate the signal by a factor of 5.
Attenuation has a few undesirable effects though.
The full range of the trigger signal is 10\,V, but the vast
majority of events will only use a small fraction of that range.
So for events that use a small amount of the available 10\,V a factor
of 5 attenuation will make the signal much small than it needs to be,
resulting in loss of information because the signal will be smaller than
the analog noise, or from the noise digitization process itself.
And for the purpose of most analyses that use the data from the CAEN
it's more important to be able to resolve a single hit than to resolve
the height of the full pulse if the pulse is very large.

So a different scheme was put in place for fitting the trigger signal
into the CAEN's available range. The trigger signal is clipped within
the first 2\,V, thereby retaining full resolution for small signals,
but losing resolution for signals that go over 2\,V.
The board that was created to perform this dynamic range reduction
was designed to optionally clip the signal or attenuate it, but
for the vast majority of data taking the signal was clipped.

The board that was designed, in part, for this purpose is the Trigger Utility
Board Mark\,-\,II (TUBII).
Beyond modifying the trigger signals for the CAEN TUBII plays a significant
role as part of the trigger and data readout systems as well.
It's significance comes primarily from the fact that it acts as an auxiliary
digital trigger board. It can receive raw-trigger pulses from the MTCA+s and apply
customizable trigger logic to them and emit it's own raw-trigger pulses which are
sent to the MTCD.
TUBII also receives the global trigger signal and produces its own trigger
word based upon which raw trigger pulses it had received. The TUBII trigger
word is synchronized with the rest of the data for each event through it's own
global trigger counter and through the SYNC/SYNC24 signals. More information about
TUBII's role in the DAQ can be found in Sec.~\ref{sec:tubii}

\subsection{Electronics Upgrades}
\label{sec:upgrades}
SN
\subsubsection{XL3}
\label{sec:xl3}
The SNO system used a centralized serial readout system, where each crate of
electronics was readout in sequence.
As part of the electronics upgrade from SNO to SNO+ this system was changed to
an asynchronous, parallel readout system.
The board responsible for this is the XL3, it hosts a Xilinx ML403
FPGA and micro-processor (XXX TODO is that the write naem??).
The ML403 reads outs out each front-end card serially and stores data
in local memory buffers before reading them out to a dedicated
data server.
\subsubsection{MTCA+}
\label{sec:mtcap}
The SNO MTCA was not expected to be able to operate stably at the expected
hit rate and occupancy of SNO+. For this reason the MTCA+ was developed,
it performs the analog multiplicity sum using a series of operational
amplifiers. The gain of the three different analog

\subsubsection{TUBII}
\label{sec:tubii}
TUBII is used as interface board for some of the detector calibration systems.
These systems emit light into the detector and usually need to be
synchronized with the trigger system. This synchronization requires
a variety of pulses and delays to be tuned to account for the time it
takes for signals and light propagate throughout the detector and DAQ
system; TUBII provides those pulses and delays.

TUBII's customizable complex trigger logic
allows it to create trigger pulses from its inputs.
The input trigger signals are fed into a Xilinx MicroZed, which is an FPGA and
micro-controller.
The MicroZed allows for nearly any logical combination of trigger signals including
using recent trigger signals to inform the current trigger logic.

Something like this is desirable for identifying and ensuring the detector will
be sensitive to time correlated events. An example of this would be that
the decay chain of Bi214 -> Po214 -> Pb212, this decay chain is referred to as BiPo214. %TODO This chain is wrong, look it up idiot
The signature of this decay is an electron from $\beta$ decay, followed, with a half
life of 4\,$\mu$s, by an $\alpha$ decay.
It's very important that the $\alpha$ decay is detected so that the $\beta$-$\alpha$
decays can be identified as likely from a BiPo. If the $\alpha$ is not observed
the $\beta$ can be mis-identified and potentially leak into a signal region.
TUBII is able to mitigate this risk by having a trigger that is particularly
sensitive to the initial $\beta$ decay and can trigger off a lower
threshold input for a short time after the $\beta$ trigger; ensuring that
the $\alpha$ is detected.

TUBII also provides general purpose and ``glue'' functionality,
facilitating different circuits from different boards in the DAQ to communicate.
An example of this is that the CAEN requires the global trigger and
other synchronization pulses be sent to it using Low-Voltage Differential
Signaling (LVDS), but the global trigger is created using Emitter Coupled
Logic (ECL).
And so TUBII provides translation between these two digital signaling protocols,
allowing the CAEN to remain synchronized.
TUBII also accepts analog signal and can apply an MTCA\,-\,like threshold
discrimination and it contains logic for creating raw-trigger pulses the
same as the MTCA+.

