
%: ----------------------- detector file header -----------------------
\chapter{Detector}

% the code below specifies where the figures are stored
\ifpdf
    \graphicspath{{detector/figures/PNG/}{detector/figures/PDF/}{detector/figures/}}
\else
    \graphicspath{{detector/figures/EPS/}{detector/figures/}}
\fi


% ----------------------------------------------------------------------
% ----------------------- Detector Chapter -----------------------------
% ----------------------------------------------------------------------

\section{The SNO+ Detector}
\subsection{The Detector in Brief}
The SNO+ detector can be mostly simply described as a large volume of some
target material that is deep underground and is observed with an array of
roughly 9500 Photomultiplier Tubes (PMTs).
Changing the target material can change what sort of physical processes is
observable with the detector.
Previously the volume was filled with ultra pure water (UPW), allowing for
observation of, among other things, high energy solar neutrinos.
For the majority of this work the target material being considered is liquid
scintillator, specifically LAB-PPO (Linear Alkylbenzene with PPO?).
The target volume is encapsulate within a 6-meter radius acrylic sphere,
which is held suspended in a large cavity filled with UPW.
Surrounding the acrylic vessel (AV) is a large array of PMTs nearly all of
which are pointing inward.
The structure holding these PMTs is referred to as the PMT Support Structure
(PSUP).
There are roughly 100 PMTs on the PSUP that point outward, toward the cavity
walls.
These outward facing tubes are called OWLs and are for the purpose of identifying
interactions that occurred outside the PSUP.
There are an additional three tubes mounted at the top of the neck of the AV,
these are referred to as NECK tubes.

\subsection{Electronics And DAQ}
The SNO+ data acquisition (DAQ) inherit much of their design and components from
SNO.
There are a few notable upgrades that were made for the purpose of handling the
higher light yield and event rate that SNO+ has compared to SNO.
The DAQ hardware can be described as two separate systems, the trigger system,
and the readout system.
The goal of the trigger system is to decide when an interesting interaction
within the detector has occurred, and to start the readout process when such an
interaction has occurred.
It does so by summing up analog pulses from each PMT.

\subsection{Scintillator}
Its just magic.

