
%: ----------------------- detector file header -----------------------
\chapter{Detector}

% the code below specifies where the figures are stored
\ifpdf
    \graphicspath{{detector/figures/PNG/}{detector/figures/PDF/}{detector/figures/}}
\else
    \graphicspath{{detector/figures/EPS/}{detector/figures/}}
\fi


% ----------------------------------------------------------------------
% ----------------------- Detector Chapter -----------------------------
% ----------------------------------------------------------------------
% 9385 PMTs comes from doing the following SQL query on DB
%select count(*) from pmt_info where (type & 1)>0 and (type & 32)=0  and (type & 128) =0 and (type & 2)>0 and (type& 64)=0 and (type & 16)=0 and (type & 8)=0;
\section{The SNO+ Detector}
\subsection{Detection Mechanism}

The primary neutrino interaction that SNO+ is sensitive to is elastic scattering
off electrons,
$\nu_{x} + e^{-} \rightarrow \nu_{x} + e^{-}$
where $x=e$,$\mu$, $\tau$.
For $x=e$ there exists a neutral current and a charged current channel,
for $x=\mu$ or $x=\tau$ there exists only the neutral current channel.
% TODO try and remember what actually I mean here
Nuclear interactions occur as well with the oxygen in the water,
however these are rare and difficult to identify, and so are ignored typically.
The elastic scattering cross-section is given by
\begin{equation}
cross section
\end{equation}
This can also be expressed as it's differential cross-section as a function
of energy
\begin{equation}
dsimga/dTe
\end{equation}
and of scattering angle
\begin{equation}
dsigma/dtheta
\end{equation}
These cross sections are shown in figure blah.
The relevant points are that the angular cross section is peaked in
the forward direction, meaning that information about the neutrino direction
is maintained in the interaction.
But the differential cross-section for recoil energy is nearly flat below the
end point. Meaning relatively little information about the incoming neutrino energy
is preserved by the interaction.
For the SNO+ water-phase the scattered electron generates light via Cherenkov radiation,
assuming it's above Cherenkov threshold for water which is $XXX$\,MeV.
If it is above the Cherenkov threshold the electron will generate photons that
travel at approximtely a 42$\deg$ angle to the direction of the electron.
The angle of travel comes directly from the speed at which electro-magnetic
signals propagate in the medium.
Figure $TODO$ shows this diagramatically, as the charged particle (in this case
an electron) distance $\ell$ from point A, to point B, the electro-magnetic wave
that was emitted at A will travel a distance $\ell \dfrac{c}{v}$, forming
a spherical wavefront at that distance.
When the electron travels another distance $\ell$ from B to C the wavefront
from A is a distance $2\ell\dfrac{c}{v}$ from A and the wavefront from B is
a distance $\ell \dfrac{c}{v}$ from B.


%%% TODO it'd be dank to have a chapter discussing Cherenkov radiation and deriving
%%% the wavelenght and angular distribution
%\includegraphs{cherenkov_radiation_cone_pictures}
The photons can the be detected by the SNO+ PMT array, and the pattern
of hits analyzed to determine the electron direction, energy, and position.

SNO+ has $XXX$ inward looking PMTs all mounted on a geodesic spere referred to as
the PMT support structure (PSUP). The PMTs are at an average radius of 8.4\,m from
the center of the detector.
Mounted on the outside of the PSUP are 90 outward looking (OWL) PMTs.
These serve to reject interactions in the outer volume from cosmic muons.
All PMTs are Hammatsu R1408 8-inch PMTs, which have typical quantum efficiency
of  $XXX$\%.

The inward looking PMTs are all housed within a plastic casette.
Each PMT is also $XXX$collared$XXX$ by an array of reflective
petals which serve to increase the effective photo-sensitive area
of the PMT.
The geometric coverage of the PMTs is $XXX$. % TODO, see if you can get the effective coverage





\subsection{The Detector in Brief}
The SNO+ detector can be mostly simply described as a large volume of some
target material that is deep underground and is observed with an array of
9385  photomultiplier tubes (PMTs).
Changing the target material can change what sort of physical processes is
observable with the detector.
The SNO detector was originally designed for a heavy-water ($\ce{^{2}H_{2}O}$ or $\ce{D_{2}O}$)
target, the upgrades for SNO+ were done in expectation of a tellurium doped
scintillator target.
For this thesis the target material is ultra-pure water (\ce{H_{2}O}).

The target volume is encapsulated within a 6-meter radius acrylic sphere,
which is held suspended in a large cavity filled with UPW.
Surrounding the acrylic vessel (AV) is an  array of inward pointing PMTs
arranged in a geodesic pattern.
The structure holding these PMTs is referred to as the PMT Support Structure
(PSUP).
There are roughly 90 PMTs on the PSUP that point outward, toward the cavity
walls.
These outward facing tubes are called OWLs and are for the purpose of tagging
interactions that occurred outside the PSUP.
There are an additional three tubes mounted at the top of the neck of the AV,
these are referred to as NECK tubes.

\subsection{Electronics And DAQ}
The SNO+ data acquisition (DAQ) inherit much of their design and components from
SNO.
There are a few notable upgrades that were made for the purpose of handling the
higher light yield and event rate that SNO+ has compared to SNO.
The DAQ hardware can be described as two separate systems, the trigger system,
and the readout system.
The goal of the trigger system is to decide when an interesting interaction
within the detector has occurred, and to start the readout process when such an
interaction has occurred.

The trigger system takes place in a few stages.
First, signals from PMTs are received by an interface board that
removes the signal from the 2\,kV power.
That signal is compared to a threshold, if the signal is over threshold a "hit"
has occurred.
At the time of the threshold crossing the following processes occur, the name for each is given
in parenthesis:
a 100ns long square pulse is created (N100), a 20ns square pulse is created (N20),
a high gain copy of the signal is created (ESUMH), a low gain copy of the signal
is created (ESUML), a linear voltage ramp begins (TAC ramp), the signal is integrated for
50\,ns with high gain (QHS), the signal is integrate for up to 400ns with high gain (QHL),
and the singal is integrated for 50ns with a low gain (QLX).

The results of the latter 4 process are recorded in analog memory cells,
if readout signal is sent to this channel within $\approx$400\,ns of the
signal crossing the values will be readout and digitized.
The first 4 signals are all combined with similar signals from
other PMT channels across the detector, \textit{i.e.} the N100 signals
from all channels with be combined and separately all the
N20 signals will be combined, \textit{etc}.
The signals are combined through analog summation, the N100 signals
from all channels are summed together, and the N20 signals from all channels are
summed together, so on and so forth.
The summing is done with three different gains on each of the
four available signals, resulting in twelve signals.
Each of the twelve signals are separately compared to a threshold;
each of the twelve thresholds are independent from each other.
This threshold is called the "trigger threshold".

The different gains are in place due to the practical difficulty of maintaining
a good signal-to-noise ratio (SNR) without limiting the range of the
system.
For example, if there exists 10\,mV of noise in the system a 20\,mV pulse
would give a 2:1 SNR, however this would mean if 5000 PMT hits occurred simultaneously
the signal would be 100\,V in size.
It is not practical to have a system with 100+\,V range and 20\,mV resolution,
so the three different gain paths allow for three different trade-offs between
SNR/resolution and range.
The highest gain signal has the best SNR, but the smallest range, and so usually
the highest gain signal has the lowest effective threshold.
The reason being that it's more important to have single hit resolution at a threshold
of 8-hits than it is at a threshold 25 hits.
The different gains on each signal are therefore labelled by their threshold (not their gain), e.g.
the high, medium and low gain paths for the N100 signal are respectively called
N100 Low (N100L), N100 Medium (N100M), and N100 High (N100H).
And although there are twelve signal-gain combinations available only seven are
used: N100-Low, N100-Med., N100-High, N20-Low, N20-Med (also called just N20), ESUMH-Low, and ESUML-Low.
Since the ESUMH and ESUML each only have one gain path available they're usually
referred to simply as ESUMH and ESUML with their gain path understood to be
the high gain path.

When a trigger signal goes over its threshold a 20\,ns digital pulse is
emitted for that signal. This pulse is called a "raw-trigger" and there is
one for each of the seven trigger signals.
Finally, each of these seven raw-trigger signals can be masked in or masked out.
If a raw-trigger is masked out, nothing happens when it fires,
If it is masked in, then a "global-trigger" signal is created.
The raw-trigger signal that caused the global trigger, as well as any other
raw-trigger signals that were high within a 20ns window of the global trigger,
are recorded and readout, this is know as the "trigger word".

The global trigger signal is sent to each channel in the detector and
causes them to digitize and readout their stored analog values (QHS, QHL, QLX and TAC).
Only channels that have been hit within the last $\approx$ 400\,ns will readout
anything, channels that were not hit will mostly ignore the global trigger signal.
Each channel, regardless of if it has received a hit or not, increments
a counter whenever it receives a global trigger signal.
For channels that were hit the value of that counter is readout
along with the hit information.
This is the only means of synchronization across the detector.

Data, once readout, is is processed by an algorithm that uses the
the value of the global trigger counter to associate PMT hits with
each other and, with their corresponding trigger word.
This forms the data that is used to reconstruct information
about the event that produced the light that was observed.

There are a few ancillary systems within the DAQ electronics
that require description. The first is the CAEN v1720, commonly
referred to as just ``CAEN'', which is a 12-bit digitizer board.
The CAEN is used to digitize and readout the trigger signals.
It has 8 available input channels that it can digitize, however,
typically only three signals are actually used: ESUMH, N20L, and
N100L. The CAEN's digitization window and sampling rate can be varied,
most commonly the digitization window is 420\,ns and the sample width
is 4\,ns. The CAEN recieves a copy of the global trigger allowing it's
data to later be synchronized with the PMT data.

The input voltage range for the CAEN is an adjustable 2\,V window.
The voltage range for the trigger signals is 10\,V wide,
necessitating some way of reducing the range of the trigger signals
before they're plugged into the CAEN.
The simplest way of reducing the voltage range is to use a voltage
divider to attenuate the signal by a factor of 5.
Attenuation has a few undesirable effects though.
The full range of the trigger signal is 10V, but the vast
majority of events will only use a small fraction of that range.
So for events that use a small amount of the available 10V a factor
of 5 attenuation will make the signal much small than it needs to be,
resulting in loss of information because the signal will be smaller than
the analog noise, or from the digitization process itself.
And for the purpose of most analyses that use the data from the CAEN
it's more important to be able to resolve a single hit than to resolve
the height of the full pulse.

So a different scheme was put in place for fitting the trigger signal
into the CAEN's available range. The trigger signal is clipped within
the first 2\,V, thereby retaining full resolution for small signals,
but losing resolution for signals that go over 2\,V.
The board that was created to perform this dynamic range reduction
was designed to optionally clip the signal or attenuate it, but
for the vast majority of data taking the signal was clipped.

The board that reduces the signal is the Trigger Utility Board Mark II (TUBII),
it plays a significant role in the SNO+ DAQ at large.
It's significance comes primarily from the fact that it acts as a secondary
digital trigger board. It can receive raw-trigger pulses and put them
through customizable trigger logic and emit it's own raw-trigger pulses.
TUBII also receives the global trigger signal and produces its own trigger
word based upon which raw trigger pulses it had received. The TUBII trigger
word is synchronized with the rest of the data for each event through it's own
global trigger counter.

TUBII is used as interface board for some of the detector calibration systems.
These systems emit light into the detector and usually need to be
synchronized with the trigger system. This synchronization requires
a variety of pulses and delays to be tuned to account for the time it
takes for signals and light propagate throughout the detector and DAQ
system; TUBII provides those pulses and delays.

As was alluded to earlier TUBII also has customizable complex trigger logic that
allows it to create trigger pulses from its inputs.
The input trigger signals are fed into a Xilinx MicroZed, which is an FPGA and
micro-controller.
The MicroZed allows for nearly any logical combination of trigger signals including
using previous trigger signals to inform the current trigger logic.

Something like this is desirable for identifying and ensuring the detector will
be sensitive to time correlated events. An example of this would be that
the decay chain of Bi214 -> Po214 -> Pb212, this decay chain is referred to as BiPo214. %TODO This chain is wrong, look it up idiot
The signature of this decay is an electron from $\beta$ decay, followed, with a half
life of 4\,$\mu$s, by an $\alpha$ decay.
It's very important that the $\alpha$ decay is detected so that the $\beta$-$\alpha$
decays can be identified as likely from a BiPo. If the $\alpha$ is not observed
the $\beta$ can be mis-identified and potentially leak into a signal region.
TUBII is able to mitigate this risk by having a trigger that is particularly
sensitive to the initial $\beta$ decay and can trigger off a lower
threshold input for a short time after the $\beta$ trigger; ensuring that
the $\alpha$ is detected.

TUBII also provides general purpose and ``glue'' functionality,
allowing different circuits from different boards in the DAQ to communicate.
An example of this is that the CAEN requires the global trigger and
other synchronization pulses be sent to it using Low-Voltage Differential
Signaling (LVDS), but the global trigger is created using Emitter Coupled
Logic (ECL).
And so TUBII provides translation between these two digital signaling protocols,
allowing the CAEN to remain synchronized.


